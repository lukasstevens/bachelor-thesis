\PassOptionsToPackage{table,svgnames,dvipsnames}{xcolor}

\usepackage[T1]{fontenc}
\usepackage[utf8]{inputenc}
\usepackage{lmodern}
\usepackage[ngerman, english]{babel}
\usepackage{hyphenat}

\usepackage[
    mode=buildnew,
    subpreambles=true
]{standalone}

\usepackage[autostyle]{csquotes}
\usepackage[%
  doi=false,
  url=false,
  style=alphabetic,
  maxnames=4,
  minnames=3,
  maxbibnames=99,
  giveninits,
  uniquename=init]{biblatex} % TODO: adapt citation style
\usepackage{graphicx}
\usepackage{scrhack} % necessary for listings package
\usepackage{listings}
\usepackage{lstautogobble}
\usepackage{tikz}
\usepackage{pgfplots}
\usepackage{pgfplotstable}
\usepackage{tabularx}
\newcolumntype{Y}{>{\centering\arraybackslash}X}
\usepackage{booktabs}
\usepackage[final]{microtype}
\usepackage{caption}
\usepackage{subcaption}
\usepackage[textsize=tiny]{todonotes}
\usepackage{float}
\usepackage[useregional]{datetime2}

\usepackage[chapter]{algorithm}
\usepackage[noend]{algpseudocode}
\makeatletter\renewcommand{\ALG@name}{Algorithmus}
\newcommand*\Let[2]{\State #1 $\gets$ #2}
\algrenewcomment[1]{\(\qquad \triangleright\) #1}
\algnewcommand{\LineComment}[1]{\State \(\triangleright\) #1}

\usepackage{amsthm}
\begingroup
    \makeatletter
    \@for\theoremstyle:=definition,remark,plain\do{%
        \expandafter\g@addto@macro\csname th@\theoremstyle\endcsname{%
            \addtolength\thm@preskip\parskip
            }%
        }
\endgroup
\usepackage{mathtools}
\usepackage{amssymb}

%\setlength{\topsep}{10pt}

\theoremstyle{plain}
\newtheorem{thm}{Satz}[section]
\newtheorem{lem}[thm]{Lemma}
\newtheorem{cor}[thm]{Korollar}

\theoremstyle{definition}
\newtheorem{defn}[thm]{Definition}
\newtheorem{bsp}[thm]{Beispiel}

\theoremstyle{definition}
\newtheorem{rem}[thm]{Bemerkung}


\bibliography{bibliography}

\setkomafont{disposition}{\normalfont\bfseries} % use serif font for headings
\linespread{1.05} % adjust line spread for mathpazo font

% Add table of contents to PDF bookmarks
\BeforeTOCHead[toc]{{\cleardoublepage\pdfbookmark[0]{\contentsname}{toc}}}

% Define TUM corporate design colors
% Taken from http://portal.mytum.de/corporatedesign/index_print/vorlagen/index_farben
\definecolor{TUMBlue}{HTML}{0065BD}
\definecolor{TUMSecondaryBlue}{HTML}{005293}
\definecolor{TUMSecondaryBlue2}{HTML}{003359}
\definecolor{TUMBlack}{HTML}{000000}
\definecolor{TUMWhite}{HTML}{FFFFFF}
\definecolor{TUMDarkGray}{HTML}{333333}
\definecolor{TUMGray}{HTML}{808080}
\definecolor{TUMLightGray}{HTML}{CCCCC6}
\definecolor{TUMAccentGray}{HTML}{DAD7CB}
\definecolor{TUMAccentOrange}{HTML}{E37222}
\definecolor{TUMAccentGreen}{HTML}{A2AD00}
\definecolor{TUMAccentLightBlue}{HTML}{98C6EA}
\definecolor{TUMAccentBlue}{HTML}{64A0C8}

% German localisation
\addto\captionsngerman{
    \renewcommand{\contentsname}{Inhaltsverzeichnis}
    \renewcommand{\listfigurename}{Abbildungsverzeichnis}
    \renewcommand{\listtablename}{Tabellenverzeichnis}
    \renewcommand{\lstlistingname}{Listing}
    \renewcommand{\lstlistlistingname}{Listingverzeichnis}
    \renewcommand{\listalgorithmname}{Algorithmenverzeichnis}
}

% Settings for lstlistings
\lstset{
	language=C++,
    basicstyle=\ttfamily,
    columns=fullflexible,
    autogobble,
    keywordstyle=\bfseries\color{TUMBlue},
    stringstyle=\color{TUMAccentGreen},
    commentstyle=\color{TUMGray},
    morecomment=[l][\color{TUMGray}]{\#},
    tabsize=2, 
    aboveskip=\intextsep,
    belowskip=\intextsep,
    frame=single
}

\usepackage[hidelinks]{hyperref} % hidelinks removes colored boxes around references and links


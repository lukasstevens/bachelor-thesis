% !TeX root = ../main.tex
% Add the above to each chapter to make compiling the PDF easier in some editors.

\chapter{Diskussion}\label{chapter:diskussion}
\section{Laufzeit}
% Verdopplung der Knotenanzahl -> ca. 10-fache Laufzeit -> $n^4$
% Exponentielle Laufzeitsteigerung sichtbar, Ansprechen der höheren Streuung bei niedrigem $\varepsilon$
% Teilanzahl, warum hoch und runter, Mehr Teile -> Feinere Schritte -> mehr Signaturen, Mehr Teile -> weniger Knoten pro Signature
% Binärbaum hohe Laufzeit, Fat Tree niedrige Laufzeit, Grad hoch -> Laufzeit runter, Verweis auf Schwierigkeitk
In Sektion~\ref{sec:cutting} wurde bewiesen, dass die Schnittphase des Algorithmus in Zeit $\bigO(\gamma^2 n^4)$ durchgeführt werden kann, wobei $\gamma \in \bigO((k/\sqrt{\eps})^{1+\ceil{1/\eps \cdot \log(1/\eps)}})$.
Wie in Sektion~\ref{sec:cutting} beschrieben, hat die Packphase eine Laufzeit von $\bigO(\gamma n ((k/\eps)^{2t} + n^2))$, wobei $t = \ceil{\log_{1 + \eps}(1/\eps)} + 1$ gilt.
Damit dominiert die Schnittphase für konstantes $k$ und konstantes $\eps$ die Laufzeit.
Dadurch ist zu erwarten, dass die Lauzeit proportional zu $n$ quartisch steigt.
Die Werte in Tabelle~\ref{tab:highnodecnt} bestätigen diese Vermutung, da eine Verdopplung der Knotenanzahl die Laufzeit um mehr als das $16$\hyp fachen erhöht.
Auch in Abbildung~\ref{fig:runnodes} wird ersichtlich, dass eine Erhöhung der Knotenanzahl von $70$ auf $130$ eine ähnliche Auswirkung auf die Laufzeit hat.
Da die Laufzeit für größere Werte von $k$ und kleinere von $\eps$ auch ansteigt, kann der Algorithmus in der Praxis nicht für größere Bäume eingesetzt werden.
Dies steht im Gegensatz zu den Heuristiken METIS und KaHIP, welche in der Lage sind Graphen mit mehr als $100000$ Knoten zu partitionieren.~\cite{KK98, SS13}

Eine weitere Beobachtung, die aus Abbildung~\ref{fig:runnodes} hervorgeht, ist die geringe Streuung der Messwerte bei vollständig balancierten Binärbäumen, die durch das in Sektion~\ref{sec:exprun} beschrieben Verfahren generiert wurden.
Da jedes Level des Binärbaums bis auf das letzte vollständig mit Knoten gefüllt ist, haben alle generierten Bäume für eine feste Knotenanzahl die gleiche Struktur.
Wie oben beschrieben dominiert für festes $k$ und $\eps$ die Schnittphase die Laufzeit.
Deshalb führen alle generierten Binärbäume zur einer ähnlichen Laufzeit, da die berechneten Signaturen in der Schnittphase allein von der Struktur des Baums abhängig sind.
Nur die Kosten der jeweiligen Signatur sind von Baum zu Baum unterschiedlich.
In Sektion~\ref{sec:exprun} wurde weiterhin angemerkt, dass Fat Trees zu einer verhältnismäßig geringen und Binärbäume zu einer verhältnismäßig hohen Laufzeit führen.
Auf dieses Verhalten wird weiter unten genauer eingegangen.

Nun wird die Auswirkung des maximalen Ungleichgewichts der Partitionierung $\eps$ auf die Laufzeit untersucht, der auschließlich einen negativen Einfluss auf die Laufzeit hat.
Während bei der Knotenanzahl $n$ steigende Werte die Laufzeit erhöhen, sind es bei $\eps$ fallende.
Die theoretische Analyse der Laufzeiten für die Schnittphase und die Packphase aus Sektion~\ref{sec:treepartitioning} zeigt, dass die Laufzeit für konstantes $n$ und $k$ mindestens exponentiell mit fallendem $\eps$ steigt.
\todo{Kann man hier wirklich exponentielle Steigerung sehen?} Die Werte für Random Attachment, Preferential Attachment und Binärbäume in Abbildung~\ref{fig:runimb} lassen für die Parameterwahl $n = 100$ und $k = 2$ zumindest eine exponentielle Entwicklung erahnen.
\todo[inline]{Auf höhere Streuung bei höherem eps eingehen? Evtl Profiling?}

Der letzte Parameter des Algorithmus ist die Anzahl der Partitionen $k$. 
Aus der Sektion~\ref{sec:treepartitioning} wissen wir, dass $k$ sowohl in die Laufzeit der Schnittphase, als auch in die Laufzeit der Packphase einfließt.
Dementsprechend sollte die Laufzeit mit steigendem $k$ zunehmen.
Die Abbildung~\ref{fig:runkparts} bestätigt diese Annahme jedoch nicht für alle Parameterkombinationen
Für $n=50$ und $\eps=1/3$ beziehungsweise $\eps=1/4$ nimmt die Laufzeit zwar zunächst zu, dann jedoch wieder ab.
Die Zunahme der Laufzeit kann mit der steigenden Anzahl der Signaturen erklärt werden.
Da die Signaturen nach Definition~\ref{defn:signature} für ein gegebenes $\eps$ immer diesselbe Länge haben, werden die Intervalleabstände der Signatur kleiner für wachsendes $k$, was die Anzahl der möglichen Signaturen erhöht.
Für höheres $k$ nimmt jedoch die maximale Größe $(1 + \eps) \ceil{n/k}$ einer Zusammenhangskomponente ab und die Intervallabstände werden so klein, dass einige Intervalle keine Zusammenhangskomponenten mehr enthalten können. 
Diese Entwicklung ist in der Tabelle~\ref{tab:ksig} zu sehen, welche die jeweiligen exklusiven oberen Schranken einer Signatur $\vec{g} = (g_1, \ldots, g_t)$ für die Parameter $n = 50$ und $\eps = 1/4$ in Abhängigkeit von $k$ abbildet.
Diese Parameter wurden auch in Abbildung~\ref{fig:runkparts} verwendet.
Auf der anderen Seite zeigt die Abbildung~\ref{fig:runkparts}, dass schon für $n=60$ der oben genannte Effekt für die gegebenen $k$ nicht mehr eintritt. 

\begin{table}
    \centering
    \begin{tabular}{l*{11}{c}}
        \toprule
        $k$ & $g_0$ & $g_1$ & $g_2$ & $g_3$ & $g_4$ & $g_5$ & $g_6$ & $g_7$ & $g_8$ \\
        \midrule
        3 & 5 & 6 & 7 & 9 & 11 & 13 & 17 & 21 & 22 \\
        4 & 4 & 5 & 6 & 7 & 8 & 10 & 13 & 16 & 17 \\
        5 & 3 & 4 & 4 & 5 & 7 & 8 & 10 & 12 & 13 \\ 
        6 & 3 & 3 & 4 & 5 & 6 & 7 & 9 & 11 & 12 \\ 
        8 & 2 & 3 & 3 & 4 & 5 & 6 & 7 & 9 & 9 \\ 
        10 & 2 & 2 & 2 & 3 & 4 & 4 & 5 & 6 & 7 \\ 
        \bottomrule
    \end{tabular}
    \caption{Exklusive obere Schranken der Signatur $\vec{g}$ für $n=50$ und $\eps=1/4$ in Abhängigkeit von $k$}\label{tab:ksig}
\end{table}

Zuletzt wird die Entwicklung der Laufzeit für steigenden Maximalgrad $\Delta$ des Baums analysiert.
Feldmann und Foschini~\cite{FF15} zeigten, dass das $(k, 1)$\hyp Partitionierungsproblem bereits ab einem Maximalgrad von $\Delta=5$ $NP$\hyp schwer ist.
Für einen Maximalgrad von $\Delta=2$ ist das Problem noch in $P$, da es sich bei dem Graphen dann um einen Pfad oder einen Kreis handelt.
Weiterhin kann der Algorithmus von MacGregor~\cite{mcg78} angepasst werden, um einen Algorithmus mit Approximationsfaktor $\bigO(\Delta \log_\Delta(n/k))$ zu erhalten.~\cite{FF15}
Daraus folgt, dass das Problem mit wachsendem Maximalgrad $\Delta$ schwieriger wird.
Die Experimente, welche durch Abbildung~\ref{fig:rundeg} visualisiert werden, zeigen jedoch, dass die Laufzeit mit wachsendem Maximalgrad abnimmt.
Um eine Intuition zu geben, warum dies der Fall ist müssen wir zuerst betrachten, wodurch die Laufzeit verringert wird.
Die Laufzeit ist dann gering, wenn wenige Signaturen berechnet werden.
An einem Knoten $v$ des Baums werden dann wenige Signaturen berechnet, wenn der Knoten keine linken Geschwisterknoten oder keine Kindknoten hat, weil nicht existierende Knoten nur die Signatur $(0, \ldots, 0)$ besitzen (siehe Sektion~\ref{sec:cutting}).
Wenn mindestens eine der Knoten nicht existiert, dann müssen die Signaturen an dem anderen Knoten nur mit der Signatur $(0, \ldots, 0)$ und allen Signaturen $\vec{e}(x)$ kombiniert werden.
Dabei $\vec{e}(x)$ die Signatur mit einer Komponente der Größe $x$ ist und $x$ läuft über alle möglichen Größen der Komponente, in der $v$ enthalten ist.
Wenn alle Knoten im Baum einen geringen Grad haben, dann ist die Anzahl der Knoten die keinen linken Geschwisterknoten hoch. 
Andererseits ist die Anzahl der Knoten die keine rechtes Kind dann hoch, wenn alle Knoten einen hohen Grad haben.
Warum also sollte die Laufzeit sinken?

Für ein intuitives Verständnis betrachten wir dazu nur den Spezialfall eines vollständig balancierten $d$\hyp ären Baums, der einen Maximalgrad von $\Delta = d + 1$ hat.
Ein $d$\hyp ärer Baum $T$ der Höhe $h$ hat $d^h$ Blätter und damit haben genau $d^h$ Knoten von $T$ kein Kind.
Um die Anzahl der Knoten zu bestimmen, die kein linkes Geschwisterkind haben betrachten wir die Level des Baums.
In jedem Level hat der erste Knoten des Levels kein linkes Geschwisterkind.
Ferner hat auch jeder $d+1$\hyp te Knoten kein Geschwisterkind, da sonst der Elternknoten mehr als $d$ Kinder haben müsste.
Da das Level $\ell$ genau $d^{\ell}$ Knoten enthält, ist die Anzahl der Knoten ohne linken Geschwisterknoten
\begin{equation*}
    \sum_{\ell=1}^{h} \frac{d^\ell}{d} = \frac{d^h - 1}{d - 1}.
\end{equation*}
Hierbei wurde die Wurzel vernachlässigt, da dort die Signaturen nur anhand der Signaturen des rechten Kinds berechnet werden.
Weiterhin ist die Anzahl der Knoten im Baum $n = 1 + d + d^2 + \ldots + d^h = (d^{h+1} - 1)/(d - 1)$ oder anders ausgedrückt, hat $T$ eine Höhe $h = \log_d((d-1)n + 1)) - 1$.
Für die Summe der Anzahl an Knoten ohne linken Geschwisterknoten und der Knoten ohne Kind erhalten wir dadurch
\begin{equation*}
    \begin{aligned}
        \frac{d^h - 1}{d - 1} + d^h & = \frac{d^{\log_d((d-1)n + 1)) - 1}}{d - 1} + d^{\log_d((d-1)n + 1)) - 1} \\
        & = \frac{(d-1)n + 1}{d(d-1)} + \frac{(d-1)n + 1}{d} = n + \frac{1}{d-1}.
    \end{aligned}
\end{equation*}
Damit müsste man $d$ minimieren?? $d$ hat dann quasi keinen Einfluss auf die Summe. Warum wird die Laufzeit mit höherem Grad geringer?

\section{Lösungsqualität}
\section{Mögliche Verbesserungen}


% !TeX root = ../main.tex
% Add the above to each chapter to make compiling the PDF easier in some editors.

\chapter{Fazit}\label{chapter:fazit}
In dieser Arbeit wurde das $k$\hyp balancierte Partitionierungsproblem betrachtet und gezeigt, dass dieses NP-schwer ist.
Anschließend wurde der die Existenz eines bikriterieller Approximationsalgorithmus für dieses Problem vorgestellt.
Aufbauend auf das theoretische Fundament wurde eine \Cpp{}\hyp Implementierung entwickelt, deren wesentliche Aspekte präsentiert wurden.
Anschließend wurde die Implementierung durch einen Vergleich mit den bekannten Graphpartitionierungsheuristiken METIS und KaFFPa experimentell evaluiert.

Wie aufgrund der theoretischen Analyse zu erwarten war, ist die Laufzeit des Algorithmus selbst für günstige Wahl der Parameter zu hoch, um ihn für größere Graphen zu verwenden.
Andererseits lieferte der Algorithmus auf kleinen Bäumen sehr gute Ergebnisse im Vergleich zu den Heuristiken.
Auf Graphen konnte sich der Algorithmus noch in manchen Fällen gegen METIS behaupten, während KaFFPa in nahezu allen Fällen bessere Resultate lieferte.
Dies untermauert den Status von KaFFPa als eine der besten Graphpartitionierungsheuristiken.

Weiterhin wurden Verbesserungsvorschläge für die Implementierung vorgestellt, deren Umsetzung in der Zukunft zu einer Verbesserung der Laufzeit des Algorithmus führen könnte.
Falls die Laufzeit ausreichend reduziert werden kann, wäre es interessant zu sehen, inwiefern der Algorithmus dieser Arbeit von den ausgefeilten Graphexpansions- und Kontraktionsmethoden der Heuristiken profitiert.
Damit könnte der Algorithmus dieser Arbeit für geeignete Werte der Parameter eine Alternative zu den bestehenden Heuristiken darstellen.

Dennoch gibt es bis jetzt noch keine Alternative für die Heuristiken, welche bei ungünstigerer Wahl der Parameter beziehungsweise für größere Graphen eine Laufzeit hat, welche in der Praxis tolerierbar ist.
Da das $(k,1+\eps)$\hyp Partitionierungsproblem in der Praxis oft auftritt, wird es von Interesse sein, die Effizienz der Approximationsalgorithmen weiter zu verbessern.
Effizientere Algorithmen, welche Garantien im Bezug auf die Lösungsqualität bieten und dabei eine akzeptable Laufzeit haben, würden eine attraktive Alternative zu den etablierten Heuristiken darstellen. 

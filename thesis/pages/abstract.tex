\chapter{Abstract}
This thesis is concerned with the implementation of a bicriteria approximation algorithm for the $k$\hyp balanced partitioning problem by Feldmann and Foschini.~\cite{FF15}
The $k$\hyp balanced partitioning problem asks to divide a graph into partitions of equal size and the objective is to minimize the total weight of edges cut by the partitioning.
The aforementioned algorithm was implemented using the programming language \Cpp{} and afterwards the implementation was compared to the graph partitioning heuristics METIS~\cite{KK98} and KaHIP~\cite{SS13}.
Evaluating the experiments shows that the algorithm is competitive with the heuristics in terms solution quality when considering small trees and graphs.
The running time, on the other hand, is high even on relatively small instances.
Additionally, the results confirm that KaHIP delivers high quality partitions.

{\let\cleardoublepage\relax \chapter{Zusammenfassung}}
Diese Bachelorarbeit beschäftigt sich mit der Implementierung eines bikriteriellen Approximationsalgorithmus für das $k$\hyp balancierte Partitionierungsproblem von Feldmann und Foschini.~\cite{FF15}
Das $k$\hyp balancierte Partitionierungsproblem fragt nach einer Partitionierung eines Eingabegraphen in Partitionen mit gleicher Größe.
Dabei minimiert eine Optimallösung dieses Problems das Gesamtgewicht der Kanten, die durch die Partitionierung geschnitten werden.
Der genannte Algorithmus wurde in der Programmiersprache \Cpp{} implementiert.
Die Implementierung wurde anschließend mit den Graphpartitionierungsheuristiken METIS~\cite{KK98} und KaHIP~\cite{SS13} verglichen.
Die Experimente zeigen, dass der Algorithmus auf kleinen Bäumen und Graphen kompetitiv mit den Heuristiken im Bezug auf die Lösungsqualität ist.
Dennoch ist die Laufzeit schon bei kleinen Probleminstanzen hoch.
Zusätzlich bestätigen die Resultate, dass KaHIP Partitionierungen mit hoher Qualität liefert.

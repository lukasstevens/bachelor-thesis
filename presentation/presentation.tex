\documentclass[12pt]{beamer}
\usetheme{metropolis}

\makeatletter
\setlength{\metropolis@progressonsectionpage@linewidth}{2pt}
\makeatother

\usepackage[utf8]{inputenc}
\usepackage[ngerman]{babel}
\usepackage{appendixnumberbeamer}
\usepackage{booktabs}
\usepackage[scale=2]{ccicons}
\usepackage{hyphenat}
\usepackage{xspace}
\usepackage{mathtools}
\usepackage{forloop}

% Define TUM corporate design colors
% Taken from http://portal.mytum.de/corporatedesign/index_print/vorlagen/index_farben
\definecolor{TUMBlue}{HTML}{0065BD}
\definecolor{TUMSecondaryBlue}{HTML}{005293}
\definecolor{TUMSecondaryBlue2}{HTML}{003359}
\definecolor{TUMBlack}{HTML}{000000}
\definecolor{TUMWhite}{HTML}{FFFFFF}
\definecolor{TUMDarkGray}{HTML}{333333}
\definecolor{TUMGray}{HTML}{808080}
\definecolor{TUMLightGray}{HTML}{CCCCC6}
\definecolor{TUMAccentGray}{HTML}{DAD7CB}
\definecolor{TUMAccentOrange}{HTML}{E37222}
\definecolor{TUMAccentGreen}{HTML}{A2AD00}
\definecolor{TUMAccentLightBlue}{HTML}{98C6EA}
\definecolor{TUMAccentBlue}{HTML}{64A0C8}

\usepackage{tikz}
\usepackage{spath3}
\usepackage{pgfplots}
\usetikzlibrary{
    decorations.pathmorphing,
    decorations.pathreplacing,
    intersections,
    matrix,
    backgrounds,
    calc
}
\usepgfplotslibrary{statistics, groupplots, fillbetween}
% Settings for pgfplots
\pgfplotsset{
    compat=newest,
    % For available color names, see http://www.latextemplates.com/svgnames-colors
    legend image code/.code={%
        \draw[#1] (0cm, -0.1cm) rectangle (0.3cm, 0.1cm);
    }
}
\tikzset{>=latex}

% Own commands
\newcommand{\eps}{\varepsilon}
\newcommand{\bigO}{\mathcal{O}}
\newcommand{\reals}{\mathbb{R}}
\newcommand{\nats}{\mathbb{N}}
\DeclarePairedDelimiter{\ceil}{\lceil}{\rceil}
\DeclarePairedDelimiter{\card}{|}{|}
\newcommand{\weight}{\omega}
\newcommand{\setunion}{\cup}
\newcommand{\Cpp}{C{}\texttt{++}}
    
\newcommand{\datadir}{}
\newcommand{\datafile}[1]{}
\newcommand{\plotdata}[2]{}

\title{Bachelorarbeit}
\subtitle{Eine experimentelle Evaluation von theoretisch fundierten Graphpartitionierungsalgorithmen}
\date{\today}
\author{Lukas Stevens}
\institute{Technische Universität München}

\begin{document}

\maketitle

\begin{frame}{Inhalt}
  \setbeamertemplate{section in toc}[sections numbered]
  \tableofcontents[hideallsubsections]
\end{frame}

\section{Das Problem}
\newcommand{\drawgraph}{%
    \draw [rotate around={-3.5:(-2.7,4.0)},line width=1.pt, dashed, TUMSecondaryBlue] (-2.7,4.0) ellipse (2.1cm and 1.6cm);
    \draw [rotate around={45.:(-6.1,1.0)},line width=1.pt, dashed, TUMAccentOrange] (-5.6,1.25) ellipse (1.8cm and 2.0cm);
    \draw [rotate around={38.40:(-1.5,-0.1)},line width=1.pt, dashed, TUMAccentGreen] (-1.4,0.0) ellipse (1.7cm and 1.9cm);
    \draw (-4.0,4.3) node[fill=TUMSecondaryBlue] (A) {};
    \draw (-2.6,3.9) node[fill=TUMSecondaryBlue] (B) {};
    \draw (-1.8,3.1) node[fill=TUMSecondaryBlue] (C) {};
    \draw (-1.2,4.1) node[fill=TUMSecondaryBlue] (D) {};
    \draw (-5.4,1.9) node[fill=TUMAccentOrange] (E) {};
    \draw (-6.2,2.8) node[fill=TUMAccentOrange] (F) {};
    \draw (-6.2,1.0) node[fill=TUMAccentOrange] (G) {};
    \draw (-4.8,0.6) node[fill=TUMAccentOrange] (H) {};
    \draw (-0.5,0.7) node[fill=TUMAccentGreen] (I) {};
    \draw (-2.1,-0.4) node[fill=TUMAccentGreen] (J) {};
    \draw (-0.4,-0.3) node[fill=TUMAccentGreen] (K) {};
    \draw (-2.2,1.1) node[fill=TUMAccentGreen] (L) {};
    \draw[line width=0.8pt] (C) -- (B) -- (D) -- (C) -- (L) -- (J) -- (K) -- (L) -- (I) -- (L) -- (H) -- (G) -- (E) -- (F) -- (G);
    \draw[line width=0.8pt] (F) -- (A);
    \draw[line width=0.8pt] (E) -- (L);
}

\begin{frame}[fragile]{$k$\hyp balancierte Partitionierung}
    \begin{columns}[onlytextwidth]
        \begin{column}{0.5\textwidth}
            \visible<+(1)->{Partitioniere Knoten in Mengen $V_1, \ldots, V_k$ mit}
            \begin{itemize}[<+(1)->]
                \item $\card{V_i} \leq \ceil{n/k}$ und
                \item Schnittkosten $C^*$ optimal. 
            \end{itemize}
        \end{column}
        \begin{column}{0.5\textwidth}
            \begin{tikzpicture}[scale=0.6, every node/.style={scale=0.6, circle}]
                \drawgraph
            \end{tikzpicture}	
        \end{column}
    \end{columns}
\end{frame}

\begin{frame}[fragile]{Motivation}
    \begin{columns}[onlytextwidth]
        \begin{column}{0.5\textwidth}
            \begin{itemize}[<+(1)->]
                \item Multi-Processing
                \item Schaltkreisentwurf
            \end{itemize}
        \end{column}
        \begin{column}{0.5\textwidth}
            \begin{tikzpicture}[scale=0.6, every node/.style={scale=0.6, circle}]
                \drawgraph
            \end{tikzpicture}	
        \end{column}
    \end{columns}
\end{frame}

\begin{frame}{Komplexität}
    \begin{itemize}[<+(1)->]
        \item $k=2$ (Minimum-Bisection-Problem) $NP$\hyp schwer
        \item $k$ beliebig
            \begin{itemize}[<+(1)->]
                \item Auf Graphen: Approximation $NP$\hyp schwer für alle $\alpha$ 
                \item Auf Bäumen: Approximation $NP$\hyp schwer für alle $\alpha = n^c$, wobei $c < 1$ konstant 
            \end{itemize}
        \item[$\Rightarrow$] Approximation des $k$\hyp balancierten Partitionierungsproblems nicht sinnvoll
    \end{itemize}
\end{frame}

\begin{frame}[fragile]{Bikriterielle Approximation}
    \begin{columns}[onlytextwidth]
        \begin{column}{0.5\textwidth}
            \begin{itemize}[<+(1)->]
                \item Schnittkosten $\alpha C^*$
                \item Relaxierung der Balanciertheit auf $\card{V_i} \leq \alert{(1 + \eps)}\ceil{n/k}$
            \end{itemize}
        \end{column}
        \begin{column}{0.5\textwidth}
            \only<1-2>{
                \begin{tikzpicture}[scale=0.6, every node/.style={scale=0.6, circle}]
                    \drawgraph
                \end{tikzpicture}	
            }
            \only<3->{
                \begin{tikzpicture}[scale=0.6, every node/.style={scale=0.6, circle}]
                    \draw [rotate around={25:(-1.8,3.5)},line width=1.pt, dashed, TUMSecondaryBlue] (-1.8,3.7) ellipse (1.2cm and 1.1cm);
                    \draw [rotate around={-15:(-6.0,2.8)},line width=1.pt, dashed, TUMAccentOrange] (-5.0,2.8) ellipse (1.8cm and 2.7cm);
                    \draw [rotate around={38.40:(-1.5,-0.1)},line width=1.pt, dashed, TUMAccentGreen] (-1.4,0.0) ellipse (1.7cm and 1.9cm);
                    \draw (-4.0,4.3) node[fill=TUMAccentOrange] (A) {};
                    \draw (-2.6,3.9) node[fill=TUMSecondaryBlue] (B) {};
                    \draw (-1.8,3.1) node[fill=TUMSecondaryBlue] (C) {};
                    \draw (-1.2,4.1) node[fill=TUMSecondaryBlue] (D) {};
                    \draw (-5.4,1.9) node[fill=TUMAccentOrange] (E) {};
                    \draw (-6.2,2.8) node[fill=TUMAccentOrange] (F) {};
                    \draw (-6.2,1.0) node[fill=TUMAccentOrange] (G) {};
                    \draw (-4.8,0.6) node[fill=TUMAccentOrange] (H) {};
                    \draw (-0.5,0.7) node[fill=TUMAccentGreen] (I) {};
                    \draw (-2.1,-0.4) node[fill=TUMAccentGreen] (J) {};
                    \draw (-0.4,-0.3) node[fill=TUMAccentGreen] (K) {};
                    \draw (-2.2,1.1) node[fill=TUMAccentGreen] (L) {};
                    \draw[line width=0.8pt] (C) -- (B) -- (D) -- (C) -- (L) -- (J) -- (K) -- (L) -- (I) -- (L) -- (H) -- (G) -- (E) -- (F) -- (G);
                    \draw[line width=0.8pt] (F) -- (A);
                    \draw[line width=0.8pt] (E) -- (L);
                \end{tikzpicture}	
            }
        \end{column}
    \end{columns}
\end{frame}


\section{Ein Bikriterieller Algorithmus}
\begin{frame}{Algorithmus von Feldmann und Foschini}
    Fast ausgewogene Partitionierung auf Bäumen mit $\alpha=1$\\
    \pause
    Zweiphasig:
    \begin{itemize}[<+(1)->]
        \item Schnittphase
        \item Packphase
    \end{itemize}
\end{frame}

\begin{frame}[fragile]{Zerlegung in Zusammenhangskomponenten}
    \begin{center}
        \begin{tikzpicture}[
            randomdraw/.style={decoration={random steps, segment length=8pt, amplitude=3pt}},
            every node/.style={scale=0.7},
            ampersand replacement=\&
        ]

            \pgfmathsetseed{23654}
            \coordinate (A) at (0.0,0.0);
            \coordinate (B) at (2.0,0.0);
            \coordinate (C) at (1.0,1.5);
            \coordinate (D) at (0.0,3.0);
            \coordinate (E) at (-1.0,1.5);
            \coordinate (F) at (-2.0,0.0);

            \node[align=left] at (1.75, 2.5) {$n=50$\\$k=4$\\$\ceil{n/k}=13$};
                
            \draw (A) -- coordinate[midway](AB) (B) -- coordinate[midway](BC) (C)
                -- coordinate[midway](CD) (D) -- coordinate[near end](DE) (E)
                -- coordinate[near start](EF) (F) -- coordinate[midway](FA) (A);
            \draw (B) -- coordinate[near end](BCM) (C);

            \visible<2->{
                \draw decorate[randomdraw]{(AB) -- (DE)};
                \draw decorate[randomdraw]{(FA) -- (C)};
            }
            \visible<3->{
                \draw decorate[randomdraw]{(EF) -- (A)};
                \draw decorate[randomdraw]{(DE) -- (CD)};
            }
            \visible<4->{
                \draw decorate[randomdraw]{(E) -- (C)};
                \draw decorate[randomdraw]{(AB) -- (BC)};
                \draw decorate[randomdraw]{(A) -- (BCM)};
            }

            \visible<5->{
                \node at (0.0,2.5) {$6$};
                \node at (0.25,1.75) {$7$};
                \node at (0.25,1.25) {$4$};
                \node at (0.75,1.0) {$3$};
                \node at (1.0,0.6125) {$3$};
                \node at (1.5,0.25) {$4$};
                \node at (0.0,0.5) {$4$};
                \node at (0.5,0.125) {$2$};
                \node at (-0.5,0.125) {$2$};
                \node at (-0.5,1.0) {$6$};
                \node at (-1.25,0.5) {$7$};
                \node at (-0.68,1.6125) {$2$};
            }
            
            %\draw[help lines] (F) grid[step=0.25] (2.0,3.0);

            \visible<6->{
		        \matrix[
                    matrix of nodes,
                    nodes={align=center,text width=0.4cm},
                ] at (0.0, -1.0) (dict) {
                    $1$ \& $2$ \& $3$ \& $4$ \& $5$ \& $6$ \& $7$ \& $8$ \& $9$ \& $10$ \& $11$ \& $12$ \& $13$ \\
                    $0$ \& $3$ \& $2$ \& $3$ \& $0$ \& $2$ \& $2$ \& $0$ \& $0$ \& $0$ \& $0$ \& $0$ \& $0$ \\
                };
                \draw (dict-1-1.south west) -- (dict-1-13.south east);
            }
        

        \end{tikzpicture}
    \end{center}
\end{frame}

\begin{frame}[fragile]{Berechnung der Repräsentanten}
    \begin{center}
        \begin{tikzpicture}[
            randomdraw/.style={decoration={random steps, segment length=8pt, amplitude=3pt}},
            every node/.style={scale=0.7}
        ]

            \pgfmathsetseed{23654}
            \begin{scope}[shift={(0.0,2.5)}, xscale=0.7, yscale=0.7, every node/.style={scale=0.55}]
                \coordinate (A) at (0.0,0.0);
                \coordinate (B) at (2.0,0.0);
                \coordinate (C) at (1.0,1.5);
                \coordinate (D) at (0.0,3.0);
                \coordinate (E) at (-1.0,1.5);
                \coordinate (F) at (-2.0,0.0);

                \node[align=left] at (1.75, 2.4) {$n=50$\\$k=4$\\$\ceil{n/k}=13$};

                \draw (A) -- coordinate[midway](AB) (B) -- coordinate[midway](BC) (C)
                    -- coordinate[midway](CD) (D) -- coordinate[near end](DE) (E)
                    -- coordinate[near start](EF) (F) -- coordinate[midway](FA) (A);
                \draw (B) -- coordinate[near end](BCM) (C);

                \visible<2->{
                    \draw[line width=1pt, TUMAccentOrange] (F) -- (B) -- (D) -- cycle;
                    \path (F) -- node[sloped, yshift=12pt, midway]{Minimale Schnittkosten} (D);
                }


                \draw decorate[randomdraw]{(AB) -- (DE)};
                \draw decorate[randomdraw]{(FA) -- (C)};
                \draw decorate[randomdraw]{(EF) -- (A)};
                \draw decorate[randomdraw]{(DE) -- (CD)};
                \draw decorate[randomdraw]{(E) -- (C)};
                \draw decorate[randomdraw]{(AB) -- (BC)};
                \draw decorate[randomdraw]{(A) -- (BCM)};

                \node at (0.0,2.5) {$6$};
                \node at (0.25,1.75) {$7$};
                \node at (0.25,1.25) {$4$};
                \node at (0.75,1.0) {$3$};
                \node at (1.0,0.57) {$3$};
                \node at (1.5,0.25) {$4$};
                \node at (0.0,0.42) {$4$};
                \node at (0.6,0.17) {$2$};
                \node at (-0.5,0.17) {$2$};
                \node at (-0.5,1.0) {$6$};
                \node at (-1.25,0.5) {$7$};
                \node at (-0.73,1.6125) {$2$};

                %\draw[help lines] (F) grid[step=0.25] (2.0,3.0);

                \matrix(dict)[
                    matrix of nodes,
                    nodes={align=center,text width=0.4cm},
                ] at (0.0, -1.0) {
                    $1$ & $2$ & $3$ & $4$ & $5$ & $6$ & $7$ & $8$ & $9$ & $10$ & $11$ & $12$ & $13$ \\
                    $0$ & $3$ & $2$ & $3$ & $0$ & $2$ & $2$ & $0$ & $0$ & $0$ & $0$ & $0$ & $0$ \\
                };
                \draw(dict-1-1.south west)--(dict-1-13.south east);
            \end{scope}

            \begin{scope}[shift={(-2.75,-0.75)}, xscale=-0.7, yscale=0.7, every node/.style={scale=0.4}]
                \coordinate (A) at (0.0,0.0);
                \coordinate (B) at (2.0,0.0);
                \coordinate (C) at (1.0,1.5);
                \coordinate (D) at (0.0,3.0);
                \coordinate (E) at (-1.0,1.5);
                \coordinate (F) at (-2.0,0.0);

                \draw (A) -- coordinate[midway](AB) (B) -- coordinate[midway](BC) (C)
                    -- coordinate[midway](CD) (D) -- coordinate[near end](DE) (E)
                    -- coordinate[near start](EF) (F) -- coordinate[midway](FA) (A);
                \draw (B) -- coordinate[near end](BCM) (C);

                \draw decorate[randomdraw]{(AB) -- (DE)};
                \draw decorate[randomdraw]{(FA) -- (C)};
                \draw decorate[randomdraw]{(EF) -- (A)};
                \draw decorate[randomdraw]{(DE) -- (CD)};
                \draw decorate[randomdraw]{(E) -- (C)};
                \draw decorate[randomdraw]{(AB) -- (BC)};
                \draw decorate[randomdraw]{(A) -- (BCM)};

                %\draw[help lines] (F) grid[step=0.25] (2.0,3.0);

            \end{scope}

            \begin{scope}[shift={(2.75,-0.75)}, xscale=0.7, yscale=0.7, every node/.style={scale=0.4}]
                \coordinate (A) at (0.0,0.0);
                \coordinate (B) at (2.0,0.0);
                \coordinate (C) at (1.0,1.5);
                \coordinate (D) at (0.0,3.0);
                \coordinate (E) at (-1.0,1.5);
                \coordinate (F) at (-2.0,0.0);

                \draw (A) -- coordinate[midway](AB) (B) -- coordinate[midway](BC) (C)
                    -- coordinate[midway](CD) (D) -- coordinate[near end](DE) (E)
                    -- coordinate[near start](EF) (F) -- coordinate[midway](FA) (A);
                \draw (B) -- coordinate[near end](BCM) (C);

                \draw decorate[randomdraw]{(AB) -- (DE)};
                \draw decorate[randomdraw]{(FA) -- (C)};
                \draw decorate[randomdraw]{(EF) -- (A)};
                \draw decorate[randomdraw]{(DE) -- (CD)};
                \draw decorate[randomdraw]{(E) -- (C)};
                \draw decorate[randomdraw]{(AB) -- (BC)};
                \draw decorate[randomdraw]{(A) -- (BCM)};

                %\draw[help lines] (F) grid[step=0.25] (2.0,3.0);

            \end{scope}

            \draw [line width=1.0pt, black] (0.0,1.0) ellipse (5.2cm and 3.8cm);

        \end{tikzpicture}
    \end{center}
\end{frame}

\begin{frame}[fragile]{Unterteilung in Äquivalenzklassen}
    \vspace{-0.5cm}
    \begin{center}
        \begin{tikzpicture}[
            randomdraw/.style={decoration={random steps, segment length=8pt, amplitude=3pt}},
            every node/.style={scale=0.7}
        ]
            \pgfmathsetseed{23654}

            \begin{scope}[shift={(0.0,3.5)}, xscale=0.5, yscale=0.5]
                \begin{scope}[shift={(0.0,2.5)}, xscale=0.7, yscale=0.7, every node/.style={scale=0.42}]
                    \coordinate (A) at (0.0,0.0);
                    \coordinate (B) at (2.0,0.0);
                    \coordinate (C) at (1.0,1.5);
                    \coordinate (D) at (0.0,3.0);
                    \coordinate (E) at (-1.0,1.5);
                    \coordinate (F) at (-2.0,0.0);

                    \draw (A) -- coordinate[midway](AB) (B) -- coordinate[midway](BC) (C)
                        -- coordinate[midway](CD) (D) -- coordinate[near end](DE) (E)
                        -- coordinate[near start](EF) (F) -- coordinate[midway](FA) (A);
                    \draw (B) -- coordinate[near end](BCM) (C);

                    \draw[line width=1pt, TUMAccentOrange] (F) -- (B) -- (D) -- cycle;
                    \path (F) -- node[sloped, yshift=9pt, xshift=-2pt, midway]{\scriptsize Minimale Schnittkosten} (D);


                    \draw decorate[randomdraw]{(AB) -- (DE)};
                    \draw decorate[randomdraw]{(FA) -- (C)};
                    \draw decorate[randomdraw]{(EF) -- (A)};
                    \draw decorate[randomdraw]{(DE) -- (CD)};
                    \draw decorate[randomdraw]{(E) -- (C)};
                    \draw decorate[randomdraw]{(AB) -- (BC)};
                    \draw decorate[randomdraw]{(A) -- (BCM)};

                    %\draw[help lines] (F) grid[step=0.25] (2.0,3.0);

                    \matrix(dict)[
                        matrix of nodes,
                        nodes={align=center,text width=0.4cm},
                    ] at (0.0, -1.0) {
                        $1$ & $2$ & $3$ & $4$ & $5$ & $6$ & $7$ & $8$ & $9$ & $10$ & $11$ & $12$ & $13$ \\
                        $0$ & $3$ & $2$ & $3$ & $0$ & $2$ & $2$ & $0$ & $0$ & $0$ & $0$ & $0$ & $0$ \\
                    };
                    \draw(dict-1-1.south west)--(dict-1-13.south east);
                \end{scope}

                \begin{scope}[shift={(-2.75,-0.75)}, xscale=-0.7, yscale=0.7, every node/.style={scale=0.4}]
                    \coordinate (A) at (0.0,0.0);
                    \coordinate (B) at (2.0,0.0);
                    \coordinate (C) at (1.0,1.5);
                    \coordinate (D) at (0.0,3.0);
                    \coordinate (E) at (-1.0,1.5);
                    \coordinate (F) at (-2.0,0.0);

                    \draw (A) -- coordinate[midway](AB) (B) -- coordinate[midway](BC) (C)
                        -- coordinate[midway](CD) (D) -- coordinate[near end](DE) (E)
                        -- coordinate[near start](EF) (F) -- coordinate[midway](FA) (A);
                    \draw (B) -- coordinate[near end](BCM) (C);

                    \draw decorate[randomdraw]{(AB) -- (DE)};
                    \draw decorate[randomdraw]{(FA) -- (C)};
                    \draw decorate[randomdraw]{(EF) -- (A)};
                    \draw decorate[randomdraw]{(DE) -- (CD)};
                    \draw decorate[randomdraw]{(E) -- (C)};
                    \draw decorate[randomdraw]{(AB) -- (BC)};
                    \draw decorate[randomdraw]{(A) -- (BCM)};

                    %\draw[help lines] (F) grid[step=0.25] (2.0,3.0);

                \end{scope}

                \begin{scope}[shift={(2.75,-0.75)}, xscale=0.7, yscale=0.7, every node/.style={scale=0.4}]
                    \coordinate (A) at (0.0,0.0);
                    \coordinate (B) at (2.0,0.0);
                    \coordinate (C) at (1.0,1.5);
                    \coordinate (D) at (0.0,3.0);
                    \coordinate (E) at (-1.0,1.5);
                    \coordinate (F) at (-2.0,0.0);

                    \draw (A) -- coordinate[midway](AB) (B) -- coordinate[midway](BC) (C)
                        -- coordinate[midway](CD) (D) -- coordinate[near end](DE) (E)
                        -- coordinate[near start](EF) (F) -- coordinate[midway](FA) (A);
                    \draw (B) -- coordinate[near end](BCM) (C);

                    \draw decorate[randomdraw]{(AB) -- (DE)};
                    \draw decorate[randomdraw]{(FA) -- (C)};
                    \draw decorate[randomdraw]{(EF) -- (A)};
                    \draw decorate[randomdraw]{(DE) -- (CD)};
                    \draw decorate[randomdraw]{(E) -- (C)};
                    \draw decorate[randomdraw]{(AB) -- (BC)};
                    \draw decorate[randomdraw]{(A) -- (BCM)};

                    %\draw[help lines] (F) grid[step=0.25] (2.0,3.0);

                \end{scope}

                \draw [line width=1.0pt, black] (0.0,1.0) ellipse (5.2cm and 3.8cm);
            \end{scope}

            % SECOND
            \begin{scope}[shift={(-3.0,0.0)}, xscale=0.5, yscale=0.5]
                \begin{scope}[shift={(0.0,2.5)}, xscale=0.7, yscale=0.7, every node/.style={scale=0.42}]
                    \coordinate (A) at (0.0,0.0);
                    \coordinate (B) at (2.0,0.0);
                    \coordinate (C) at (1.0,1.5);
                    \coordinate (D) at (0.0,3.0);
                    \coordinate (E) at (-1.0,1.5);
                    \coordinate (F) at (-2.0,0.0);

                    \draw (A) -- coordinate[midway](AB) (B) -- coordinate[midway](BC) (C)
                        -- coordinate[midway](CD) (D) -- coordinate[near end](DE) (E)
                        -- coordinate[near start](EF) (F) -- coordinate[midway](FA) (A);
                    \draw (B) -- coordinate[near end](BCM) (C);

                    \draw[line width=1pt, TUMAccentOrange] (F) -- (B) -- (D) -- cycle;
                    \path (F) -- node[sloped, yshift=9pt, xshift=-2pt, midway]{\scriptsize Minimale Schnittkosten} (D);


                    \draw decorate[randomdraw]{(AB) -- (DE)};
                    \draw decorate[randomdraw]{(FA) -- (C)};
                    \draw decorate[randomdraw]{(D) -- (A)};
                    \draw decorate[randomdraw]{(A) -- (EF)};
                    \draw decorate[randomdraw]{(EF) -- (FA)};
                    \draw decorate[randomdraw]{(DE) -- (CD)};
                    \draw decorate[randomdraw]{(CD) -- (EF)};
                    \draw decorate[randomdraw]{(AB) -- (BC)};
                    \draw decorate[randomdraw]{(A) -- (BCM)};

                    %\draw[help lines] (F) grid[step=0.25] (2.0,3.0);

                    \matrix(dict)[
                        matrix of nodes,
                        nodes={align=center,text width=0.4cm},
                    ] at (0.0, -1.0) {
                        $1$ & $2$ & $3$ & $4$ & $5$ & $6$ & $7$ & $8$ & $9$ & $10$ & $11$ & $12$ & $13$ \\
                        $2$ & $5$ & $4$ & $2$ & $2$ & $0$ & $1$ & $0$ & $1$ & $0$ & $0$ & $0$ & $0$ \\
                    };
                    \draw(dict-1-1.south west)--(dict-1-13.south east);
                \end{scope}

                \begin{scope}[shift={(-2.75,-0.75)}, xscale=-0.7, yscale=0.7, every node/.style={scale=0.4}]
                    \coordinate (A) at (0.0,0.0);
                    \coordinate (B) at (2.0,0.0);
                    \coordinate (C) at (1.0,1.5);
                    \coordinate (D) at (0.0,3.0);
                    \coordinate (E) at (-1.0,1.5);
                    \coordinate (F) at (-2.0,0.0);

                    \draw (A) -- coordinate[midway](AB) (B) -- coordinate[midway](BC) (C)
                        -- coordinate[midway](CD) (D) -- coordinate[near end](DE) (E)
                        -- coordinate[near start](EF) (F) -- coordinate[midway](FA) (A);
                    \draw (B) -- coordinate[near end](BCM) (C);

                    %\draw[help lines] (F) grid[step=0.25] (2.0,3.0);
                    \draw decorate[randomdraw]{(AB) -- (DE)};
                    \draw decorate[randomdraw]{(FA) -- (C)};
                    \draw decorate[randomdraw]{(EF) -- (A)};
                    \draw decorate[randomdraw]{(EF) -- (FA)};
                    \draw decorate[randomdraw]{(DE) -- (CD)};
                    \draw decorate[randomdraw]{(E) -- (C)};
                    \draw decorate[randomdraw]{(AB) -- (BC)};
                    \draw decorate[randomdraw]{(A) -- (BCM)};
                    \draw decorate[randomdraw]{(D) -- (A)};

                \end{scope}

                \begin{scope}[shift={(2.75,-0.75)}, xscale=0.7, yscale=0.7, every node/.style={scale=0.4}]
                    \coordinate (A) at (0.0,0.0);
                    \coordinate (B) at (2.0,0.0);
                    \coordinate (C) at (1.0,1.5);
                    \coordinate (D) at (0.0,3.0);
                    \coordinate (E) at (-1.0,1.5);
                    \coordinate (F) at (-2.0,0.0);

                    \draw (A) -- coordinate[midway](AB) (B) -- coordinate[midway](BC) (C)
                        -- coordinate[midway](CD) (D) -- coordinate[near end](DE) (E)
                        -- coordinate[near start](EF) (F) -- coordinate[midway](FA) (A);
                    \draw (B) -- coordinate[near end](BCM) (C);

                    %\draw[help lines] (F) grid[step=0.25] (2.0,3.0);
                    \draw decorate[randomdraw]{(AB) -- (DE)};
                    \draw decorate[randomdraw]{(FA) -- (C)};
                    \draw decorate[randomdraw]{(EF) -- (A)};
                    \draw decorate[randomdraw]{(EF) -- (FA)};
                    \draw decorate[randomdraw]{(DE) -- (CD)};
                    \draw decorate[randomdraw]{(D) -- (A)};
                    \draw decorate[randomdraw]{(E) -- (C)};
                    \draw decorate[randomdraw]{(AB) -- (BC)};
                    \draw decorate[randomdraw]{(BCM) -- (A)};

                \end{scope}

                \draw [line width=1.0pt, black] (0.0,1.0) ellipse (5.2cm and 3.8cm);
            \end{scope}

            % THIRD
            \begin{scope}[shift={(3.0,0.0)}, xscale=0.5, yscale=0.5]
                \begin{scope}[shift={(0.0,2.5)}, xscale=0.7, yscale=0.7, every node/.style={scale=0.42}]
                    \coordinate (A) at (0.0,0.0);
                    \coordinate (B) at (2.0,0.0);
                    \coordinate (C) at (1.0,1.5);
                    \coordinate (D) at (0.0,3.0);
                    \coordinate (E) at (-1.0,1.5);
                    \coordinate (F) at (-2.0,0.0);

                    \draw (A) -- coordinate[midway](AB) (B) -- coordinate[midway](BC) (C)
                        -- coordinate[midway](CD) (D) -- coordinate[near end](DE) (E)
                        -- coordinate[near start](EF) (F) -- coordinate[midway](FA) (A);
                    \draw (B) -- coordinate[near end](BCM) (C);

                    \draw[line width=1pt, TUMAccentOrange] (F) -- (B) -- (D) -- cycle;
                    \path (F) -- node[sloped, yshift=9pt, xshift=-2pt, midway]{\scriptsize Minimale Schnittkosten} (D);


                    \draw decorate[randomdraw]{(F) -- (C)};
                    \draw decorate[randomdraw]{(DE) -- (A)};
                    \draw decorate[randomdraw]{(C) -- (A)};

                    %\draw[help lines] (F) grid[step=0.25] (2.0,3.0);

                    \matrix(dict)[
                        matrix of nodes,
                        nodes={align=center,text width=0.4cm},
                    ] at (0.0, -1.0) {
                        $1$ & $2$ & $3$ & $4$ & $5$ & $6$ & $7$ & $8$ & $9$ & $10$ & $11$ & $12$ & $13$ \\
                        $0$ & $0$ & $0$ & $0$ & $0$ & $0$ & $0$ & $1$ & $0$ & $2$ & $2$ & $0$ & $0$ \\
                    };
                    \draw(dict-1-1.south west)--(dict-1-13.south east);
                \end{scope}

                \begin{scope}[shift={(-2.75,-0.75)}, xscale=-0.7, yscale=0.7, every node/.style={scale=0.4}]
                    \coordinate (A) at (0.0,0.0);
                    \coordinate (B) at (2.0,0.0);
                    \coordinate (C) at (1.0,1.5);
                    \coordinate (D) at (0.0,3.0);
                    \coordinate (E) at (-1.0,1.5);
                    \coordinate (F) at (-2.0,0.0);

                    \draw (A) -- coordinate[midway](AB) (B) -- coordinate[midway](BC) (C)
                        -- coordinate[midway](CD) (D) -- coordinate[near end](DE) (E)
                        -- coordinate[near start](EF) (F) -- coordinate[midway](FA) (A);
                    \draw (B) -- coordinate[near end](BCM) (C);

                    \draw decorate[randomdraw]{(F) -- (C)};
                    \draw decorate[randomdraw]{(DE) -- (A)};
                    \draw decorate[randomdraw]{(C) -- (A)};

                    %\draw[help lines] (F) grid[step=0.25] (2.0,3.0);

                \end{scope}

                \begin{scope}[shift={(2.75,-0.75)}, xscale=0.7, yscale=0.7, every node/.style={scale=0.4}]
                    \coordinate (A) at (0.0,0.0);
                    \coordinate (B) at (2.0,0.0);
                    \coordinate (C) at (1.0,1.5);
                    \coordinate (D) at (0.0,3.0);
                    \coordinate (E) at (-1.0,1.5);
                    \coordinate (F) at (-2.0,0.0);

                    \draw (A) -- coordinate[midway](AB) (B) -- coordinate[midway](BC) (C)
                        -- coordinate[midway](CD) (D) -- coordinate[near end](DE) (E)
                        -- coordinate[near start](EF) (F) -- coordinate[midway](FA) (A);
                    \draw (B) -- coordinate[near end](BCM) (C);

                    \draw decorate[randomdraw]{(C) -- (A)};
                    \draw decorate[randomdraw]{(F) -- (C)};
                    \draw decorate[randomdraw]{(A) -- (DE)};

                    %\draw[help lines] (F) grid[step=0.25] (2.0,3.0);

                \end{scope}

                \draw [line width=1.0pt, black] (0.0,1.0) ellipse (5.2cm and 3.8cm);
            \end{scope}
        \end{tikzpicture}
    \end{center}
\end{frame}

\begin{frame}[fragile]{Packen der Zusammenhangskomponenten}
    \begin{center}
        \begin{tikzpicture}[
            randomdraw/.style={decoration={random steps, segment length=8pt, amplitude=3pt}},
            every node/.style={scale=0.8},
            ampersand replacement=\&
        ]

            \pgfmathsetseed{23654}
            \coordinate (A) at (0.0,0.0);
            \coordinate (B) at (2.0,0.0);
            \coordinate (C) at (1.0,1.5);
            \coordinate (D) at (0.0,3.0);
            \coordinate (E) at (-1.0,1.5);
            \coordinate (F) at (-2.0,0.0);

            \node[align=left] at (1.75, 2.5) {$n=50$};
                
            \draw (A) -- coordinate[midway](AB) (B) -- coordinate[midway](BC) (C)
                -- coordinate[midway](CD) (D) -- coordinate[near end](DE) (E)
                -- coordinate[near start](EF) (F) -- coordinate[midway](FA) (A);
            \draw (B) -- coordinate[near end](BCM) (C);

            \draw[name path=ABDE, save path=\pathABDE] decorate[randomdraw]{(AB) -- (DE)};
            \draw[name path=FAC, save path=\pathFAC] decorate[randomdraw]{(FA) -- (C)};
            \draw[name path=EFA, save path=\pathEFA] decorate[randomdraw]{(EF) -- (A)};
            \draw[name path=DECD, save path=\pathDECD] decorate[randomdraw]{(DE) -- (CD)};
            \draw[name path=EC, save path=\pathEC] decorate[randomdraw]{(E) -- (C)};
            \draw[name path=ABBC, save path=\pathABBC] decorate[randomdraw]{(AB) -- (BC)};
            \draw[name path=ABCM, save path=\pathABCM] decorate[randomdraw]{(A) -- (BCM)};

            \path[name path=ECABDERTWO]
                [intersection segments={of=EC and ABDE, sequence={R1}}];
            \path[name path=EFAFACRTWO]
                [intersection segments={of=EFA and FAC, sequence={R2}}];
            \path[name path=PREVLONE]
                [intersection segments={of=ECABDERTWO and EFAFACRTWO, sequence={L1}}];

            \begin{scope}[on background layer]
                \visible<2->{
                    \fill[fill=TUMAccentBlue, intersection segments={
                        of=ABCM and ABDE, sequence={R1--L1[reverse]}}] -- cycle;
                    
                    \fill[fill=TUMAccentBlue]
                        [intersection segments={of=ECABDERTWO and EFAFACRTWO, sequence={L2[reverse]--R2}}]
                        [intersection segments={of=ABDE and EC, sequence={R2[reverse]}}]
                        ;

                    \fill[fill=TUMAccentBlue]
                        [intersection segments={of=PREVLONE and ABCM, sequence={R2[reverse]--L2}}]
                        [intersection segments={of=ECABDERTWO and EFAFACRTWO, sequence={R2}}] -- (BCM)
                        ;

                    \fill[fill=TUMAccentBlue]
                        [intersection segments={of=ABBC and ABBC, sequence={L*[reverse]}}]
                        [intersection segments={of=ECABDERTWO and ABCM, sequence={L1--R2}}] -- (BC);
                }

                \visible<3->{
                    \fill[fill=TUMAccentOrange, intersection segments={
                        of=ABDE and EC, sequence=L2--R1}];

                    \fill[fill=TUMAccentOrange, intersection segments={
                        of=FAC and EFA, sequence={L1--R2}}] -- cycle;

                    \fill[fill=TUMAccentOrange]
                        [intersection segments={of=ECABDERTWO and EFAFACRTWO, sequence={R1}}] -- (A)
                        [intersection segments={of=PREVLONE and ABCM, sequence={L2[reverse]--R1[reverse]}}]
                        [intersection segments={of=FAC and EFA, sequence={R2[reverse]}}]
                        ;

                }
                \makeatletter
                \pgfsyssoftpath@setcurrentpath{\pathABBC} \visible<3->{\filldraw[draw, fill=TUMAccentOrange] -- (B) -- cycle;}
                \makeatother


                \visible<4->{
                    \fill[fill=TUMAccentGreen, intersection segments={
                        of=EFA and FAC, sequence={L1--R1[reverse]}}] -- (F) -- cycle;

                    \fill[fill=TUMAccentGreen]
                        [intersection segments={of=EC and ABDE, sequence={L1}}]
                        [intersection segments={of=ECABDERTWO and EFAFACRTWO, sequence={L2[reverse] -- R1[reverse]}}] -- (E)
                        [intersection segments={of=EFA and FAC, sequence={L1[reverse]}}] -- (E)
                        ;
                }

                \visible<5->{
                    \fill[fill=TUMGray]
                        [intersection segments={of=EC and ABDE, sequence={L2[reverse] -- R2}}]
                        [intersection segments={of=DECD and DECD, sequence={L*}}] -- (C);

                }
                \makeatletter
                \pgfsyssoftpath@setcurrentpath{\pathDECD} \visible<5->{\filldraw[draw, fill=TUMGray] -- (D) -- cycle;}
                \makeatother

            \end{scope}

            \node at (0.0,2.5) {$6$};
            \node at (0.25,1.75) {$7$};
            \node at (0.25,1.25) {$4$};
            \node at (0.75,1.0) {$3$};
            \node at (1.0,0.6125) {$3$};
            \node at (1.5,0.25) {$4$};
            \node at (0.0,0.5) {$4$};
            \node at (0.5,0.125) {$2$};
            \node at (-0.5,0.125) {$2$};
            \node at (-0.5,1.0) {$6$};
            \node at (-1.25,0.5) {$7$};
            \node at (-0.68,1.6125) {$2$};
            
            %\draw[help lines] (F) grid[step=0.25] (2.0,3.0);

		    \matrix[
                matrix of nodes,
                nodes={align=center,text width=0.4cm},
            ] at (-5.5, 4.0) (dict) {
                $1$ \& $2$ \& $3$ \& $4$ \& $5$ \& $6$ \& $7$ \& $8$ \& $9$ \& $10$ \& $11$ \& $12$ \& $13$ \\
                $0$ \& $3$ \& $2$ \& $3$ \& $0$ \& $2$ \& $2$ \& $0$ \& $0$ \& $0$ \& $0$ \& $0$ \& $0$ \\
            };
            \draw (dict-1-1.south west) -- (dict-1-13.south east);
			
			\foreach \bin/\x in {ONE/-8.5,TWO/-7,THREE/-5.5,FOUR/-4}{
				\coordinate (\bin{}BL) at (\x,0);
				\coordinate (\bin{}BR) at ($ (\x,0) + (0.9,0.0) $);
				\coordinate (\bin{}TL) at ($ (\x,0) + (0.0,2.6) $);
				\coordinate (\bin{}TR) at ($ (\x,0) + (0.9,2.6) $);

                \draw[line width=1pt] (\bin{}TL) -- (\bin{}BL) -- (\bin{}BR) -- (\bin{}TR);
			}

            \begin{scope}[on background layer]
                \foreach \bin/\y/\s in {ONE/0/0.4,ONE/0.4/0.6,ONE/1.0/0.6,ONE/1.6/0.8}{
                    \visible<2->{\filldraw[draw=white, fill=TUMAccentBlue] ($ (\bin{}BL) + (0.0,\y) $) rectangle ($ (\bin{}BL) + (0.9,\y+\s) $);}
			    }

                \foreach \bin/\y/\s in {TWO/0/0.4,TWO/0.4/0.4,TWO/0.8/0.8,TWO/1.6/0.8}{
                    \visible<3->{\filldraw[draw=white, fill=TUMAccentOrange] ($ (\bin{}BL) + (0.0,\y) $) rectangle ($ (\bin{}BL) + (0.9,\y+\s) $);}
			    }

                \foreach \bin/\y/\s in {THREE/0/1.2,THREE/1.2/1.4}{
                    \visible<4->{\filldraw[draw=white, fill=TUMAccentGreen] ($ (\bin{}BL) + (0.0,\y) $) rectangle ($ (\bin{}BL) + (0.9,\y+\s) $);}
			    }

                \foreach \bin/\y/\s in {FOUR/0/1.2,FOUR/1.2/1.4}{
                    \visible<5->{\filldraw[draw=white, fill=TUMGray] ($ (\bin{}BL) + (0.0,\y) $) rectangle ($ (\bin{}BL) + (0.9,\y+\s) $);}
			    }
            \end{scope}

            \foreach \bin in {ONE,TWO,THREE,FOUR}{ 
                \path (\bin{}TL) -- node[midway] (\bin{}TM) {} (\bin{}TR);
            }

            \foreach \bin/\s/\vis in {ONE/2/2,ONE/3/2,ONE/3/2,ONE/4/2,TWO/4/3,TWO/4/3,TWO/2/3,TWO/2/3,THREE/7/4,THREE/6/4,FOUR/7/5,FOUR/6/5}{
                \visible<\vis->{\draw (dict-2-\s.south) edge[->, line width=1pt, out=270, in=90] (\bin{}TM.north);}
            }

            \path (TWO{}BR.south) -- node[midway, yshift=-0.5cm] {$k=4$} (THREE{}BL.south);
            \draw[<-, line width=1pt] ($ (FOUR{}TR.east) + (0.1,0.0) $) -- ($ (FOUR{}TR.east) + (0.5,0.0) $) node[xshift=1cm] {$\ceil{n/k}=13$};
        \end{tikzpicture}
    \end{center}
\end{frame}

\begin{frame}[fragile]{Packen der Zusammenhangskomponenten}
    \begin{tikzpicture}[
            randomdraw/.style={decoration={random steps, segment length=8pt, amplitude=3pt}},
            every node/.style={scale=0.8},
        ]
        \pgfmathsetseed{2314}
        \coordinate (A) at (0.0,0.0);
        \coordinate (B) at (2.0,0.0);
        \coordinate (C) at (1.0,1.5);
        \coordinate (D) at (0.0,3.0);
        \coordinate (E) at (-1.0,1.5);
        \coordinate (F) at (-2.0,0.0);

        \node at (-0.75,0.25) {$10$};
        \node at (1.0,0.5) {$11$};
        \node at (0,1.75) {$11$};
        \node at (-0.9,1) {$10$};
        \node at (0,0.75) {$8$};

        \draw (A) -- coordinate[midway](AB) (B) -- coordinate[midway](BC) (C)
            -- coordinate[midway](CD) (D) -- coordinate[near end](DE) (E)
            -- coordinate[near start](EF) (F) -- coordinate[midway](FA) (A);
        \draw (B) -- coordinate[near end](BCM) (C);


        \draw[name path=FC] decorate[randomdraw]{(F) -- (C)};
        \draw[name path=DEA] decorate[randomdraw]{(DE) -- (A)};
        \draw[name path=CA] decorate[randomdraw]{(C) -- (A)};

        %\draw[help lines] (F) grid[step=0.25] (2.0,3.0);

        \begin{scope}[on background layer]

            \visible<2->{\fill[fill=TUMAccentBlue]
                [intersection segments={of=FC and DEA, sequence={R1--L2}}] -- (D) -- (D)
                ;
            }

            \visible<3->{\fill[fill=TUMAccentOrange]
                [intersection segments={of=FC and DEA, sequence={L1--R2}}] -- (F)
                ;
            }

            \visible<4->{\fill[fill=TUMAccentGreen]
                [intersection segments={of=CA and CA, sequence={L*}}] -- (B) -- (C)
                ;
            }
            \visible<5->{\fill[fill=TUMGray]
                [intersection segments={of=FC and DEA, sequence={R1--L1[reverse]}}] -- (DE)
                ;
            }
        \end{scope}

        \matrix(dict)[
            matrix of nodes,
            nodes={align=center,text width=0.4cm},
        ] at (-5.5, 4.0) {
            $1$ & $2$ & $3$ & $4$ & $5$ & $6$ & $7$ & $8$ & $9$ & $10$ & $11$ & $12$ & $13$ \\
            $0$ & $0$ & $0$ & $0$ & $0$ & $0$ & $0$ & $1$ & $0$ & $2$ & $2$ & $0$ & $0$ \\
        };

        \draw(dict-1-1.south west)--(dict-1-13.south east);

	    \foreach \bin/\x in {ONE/-8.5,TWO/-7,THREE/-5.5,FOUR/-4}{
		\coordinate (\bin{}BL) at (\x,0);
		\coordinate (\bin{}BR) at ($ (\x,0) + (0.9,0.0) $);
		\coordinate (\bin{}TL) at ($ (\x,0) + (0.0,2.6) $);
		\coordinate (\bin{}TR) at ($ (\x,0) + (0.9,2.6) $);

        \draw[line width=1pt] (\bin{}TL) -- (\bin{}BL) -- (\bin{}BR) -- (\bin{}TR);
	    }

        \begin{scope}[on background layer]
            \foreach \bin/\y/\s in {ONE/0/2.2}{
                \visible<2->{\filldraw[draw=white, fill=TUMAccentBlue] ($ (\bin{}BL) + (0.0,\y) $) rectangle ($ (\bin{}BL) + (0.9,\y+\s) $);}
	        }

            \foreach \bin/\y/\s in {TWO/0/2.0}{
                \visible<3->{\filldraw[draw=white, fill=TUMAccentOrange] ($ (\bin{}BL) + (0.0,\y) $) rectangle ($ (\bin{}BL) + (0.9,\y+\s) $);}
	        }

            \foreach \bin/\y/\s in {THREE/0/2.2}{
                \visible<4->{\filldraw[draw=white, fill=TUMAccentGreen] ($ (\bin{}BL) + (0.0,\y) $) rectangle ($ (\bin{}BL) + (0.9,\y+\s) $);}
	        }

            \foreach \bin/\y/\s in {FOUR/0/2.0}{
                \visible<5->{\filldraw[draw=white, fill=TUMGray] ($ (\bin{}BL) + (0.0,\y) $) rectangle ($ (\bin{}BL) + (0.9,\y+\s) $);}
	        }
        \end{scope}

        \foreach \bin in {ONE,TWO,THREE,FOUR}{ 
            \path (\bin{}TL) -- node[midway] (\bin{}TM) {} (\bin{}TR);
        }

        \foreach \bin/\s/\vis in {ONE/11/2,TWO/10/3,THREE/11/4,FOUR/10/5}{
            \visible<\vis->{\draw (dict-2-\s.south) edge[->, line width=1pt, out=270, in=90] (\bin{}TM.north);}
        }

        \path (TWO{}BR.south) -- node[midway, yshift=-0.5cm] {$k=4$} (THREE{}BL.south);
        \draw[<-, line width=1pt] ($ (FOUR{}TR.east) + (0.1,0.0) $) -- ($ (FOUR{}TR.east) + (0.5,0.0) $) node[xshift=1cm] {$\ceil{n/k}=13$};
    \end{tikzpicture}
\end{frame}

\begin{frame}[standout]
    Problem: Exponentielle Anzahl an Äquivalenzklassen in $n$
\end{frame}

\begin{frame}[fragile]{Gröbere Signaturen}
    \begin{center}
        \begin{tikzpicture}[
	    	seps/.style={
	    		line width=1.5pt,
	    		color=TUMAccentOrange
	    	}
	    ]
            \matrix(dict)[
                matrix of nodes,
                nodes={align=center,text width=0.32cm},
            ] at (0, 0) {
                $1$ & $2$ & $3$ & $4$ & $5$ & $6$ & $7$ & $8$ & $9$ & $10$ & $11$ & $12$ & $13$ & $14$ & $15$ & $16$ & $17$ & $\cdots$ \\
                $1$ & $0$ & $2$ & $3$ & $1$ & $4$ & $1$ & $0$ & $0$ & $2$ & $3$ & $0$ & $4$ & $0$ & $5$ & $2$ & $1$ & $\cdots$ \\
            };
            \draw(dict-1-1.south west)--(dict-1-17.south east) node[xshift=0.3cm, yshift=-0.02cm] {$\cdots$} ($ (dict-1-17.south) + (0.8,0.0) $);

            \foreach \n/\label in {5/{\eps \ceil{n/k}},10/{{(1+\eps)}^2 \cdot \eps \ceil{n/k}},15/{{(1+\eps)}^3 \cdot \eps \ceil{n/k}}}{
                \visible<2->{\draw[seps] (dict-2-\n.south east) -- (dict-1-\n.north east) node[yshift=0.5cm] {$\label$};}
            }
            \visible<2->{\draw[seps] (dict-2-7.south east) -- ($ (dict-1-7.north east) + (0.0,0.7) $) node [yshift=0.5cm] {$(1+\eps) \cdot \eps \ceil{n/k}$};}

            \foreach \from/\to/\tob/\sum in {1/5/6/7,6/7/8/5,8/10/11/2,11/15/16/12}{
                \visible<3->{\draw[line width=1pt, decoration={brace,mirror,raise=3pt},decorate] (dict-2-\from .south west) -- coordinate[midway, yshift=-0.6cm] (\from-\to)  (dict-2-\to .south east);} 
                \visible<4->{\node at (\from-\to) (\from-\to-node) {$[\from,\tob)$} ;}
                \visible<4->{\node at (\from-\to) [yshift=-0.6cm] {$\sum$};}
            }
            \visible<4->{\draw (1-5-node.south west) -- (11-15-node.south east) node[xshift=0.3cm, yshift=-0.02cm] {$\cdots$};}

        \end{tikzpicture}
    \end{center}
\end{frame}

\begin{frame}{Reduktion der Signaturanzahl}
    \begin{itemize}[<+(1)->]
        \item Anzahl der gröberen Signaturen: {\Large \[ \bigO\left(n{\left(k/\sqrt{\eps}\right)}^{1+\frac{1}{\eps} \log(\frac{1}{\eps})}\right) \]}
        \item[$\Rightarrow$] Anzahl der Signaturen polynomiell für konstantes $\eps$\\
        \item[$\Rightarrow$] Polynomielle Anzahl an Äquivalenzklassen
    \end{itemize}
\end{frame}

\begin{frame}{Algorithmus}
    \begin{enumerate}[<+(1)->]
            \item Berechnung der Äquivalenzklassen mit dynamischer Programmierung
            \item Packen der Signaturen in Behälter der Größe $(1+\eps) \ceil{n/k}$
            \item Wähle zulässigen Repräsentanten mit kleinsten Schnittkosten
    \end{enumerate}
    \pause
    Approximationsfaktor: 1
\end{frame}

\begin{frame}{Verallgemeinerung auf Graphen}
    \pause
    Algorithmus von Räcke:
    \begin{itemize}[<+(1)->]
        \item \textbf{Eingabe:} Graph
        \item \textbf{Ausgabe:} Konvexkombination von Bäumen
        \item Anzahl der Bäume polynomiell
        \item Mindestens ein Baum approximiert Schnittkosten mit $\bigO(\log n)$
    \end{itemize}
\end{frame}

\begin{frame}[fragile]{Verallgemeinerung auf Graphen}
    \pgfmathsetseed{1}
    \begin{columns}[onlytextwidth]
        \begin{column}{0.175\textwidth}
            \begin{tikzpicture}[scale=0.4, every node/.style={scale=0.5, circle, fill=TUMGray}]
                \draw (-4.0,4.3) node (A) {};
                \draw (-2.6,3.9) node (B) {};
                \draw (-1.8,3.1) node (C) {};
                \draw (-1.2,4.1) node (D) {};
                \draw (-5.4,1.9) node (E) {};
                \draw (-6.2,2.8) node (F) {};
                \draw (-6.2,1.0) node (G) {};
                \draw (-4.8,0.6) node (H) {};
                \draw (-0.5,0.7) node (I) {};
                \draw (-2.1,-0.4) node (J) {};
                \draw (-0.4,-0.3) node (K) {};
                \draw (-2.2,1.1) node (L) {};
                \draw[line width=0.8pt] (C) -- (B) -- (D) -- (C) -- (L) -- (J) -- (K) -- (L) -- (I) -- (L) -- (H) -- (G) -- (E) -- (F) -- (G);
                \draw[line width=0.8pt] (F) -- (A);
                \draw[line width=0.8pt] (E) -- (L);
            \end{tikzpicture}	
        \end{column}

        \begin{column}{0.1\textwidth}
            \begin{tikzpicture}
                \draw[->,line width=1pt] (0,0) -- node[midway, above, scale=0.6, yshift=3pt] {Dekomp.} (1,0);
            \end{tikzpicture}
        \end{column}
        
        \begin{column}{0.2\textwidth}

            \begin{minipage}[0.4\textheight]{0.1\textwidth}
                \begin{tikzpicture}[scale=0.3, every node/.style={scale=0.4, circle,fill=TUMGray}, draw=TUMGray]
                    \draw (-4.0,4.3) node[fill=TUMSecondaryBlue] (A) {};
                    \draw (-2.6,3.9) node[fill=TUMSecondaryBlue] (B) {};
                    \draw (-1.8,3.1) node[fill=TUMAccentGreen] (C) {};
                    \draw (-1.2,4.1) node[fill=TUMSecondaryBlue] (D) {};
                    \draw (-5.4,1.9) node[fill=TUMSecondaryBlue] (E) {};
                    \draw (-6.2,2.8) node[fill=TUMAccentOrange] (F) {};
                    \draw (-6.2,1.0) node[fill=TUMAccentOrange] (G) {};
                    \draw (-4.8,0.6) node[fill=TUMAccentOrange] (H) {};
                    \draw (-0.5,0.7) node[fill=TUMAccentGreen] (I) {};
                    \draw (-2.1,-0.4) node[fill=TUMAccentGreen] (J) {};
                    \draw (-0.4,-0.3) node[fill=TUMAccentGreen] (K) {};
                    \draw (-2.2,1.1) node[fill=TUMAccentOrange] (L) {};

                    \node at (-5,3) (AF) {};
                    \node at (-3.5,1.75)  (MID) {};
                    \node at (-2.75,1.75) (NB) {};
                    \node at (-1.5,1.5) (RIGHT) {};
                    \node at (-1,2.5) (CD) {};
                    \draw[line width=1.5+rand] (G) -- node[midway] (GH) {} (H);
                    \draw[line width=1.5+rand] (A) -- (AF) -- (F);
                    \draw[line width=1.5+rand] (GH) -- (MID) -- (AF);
                    \draw[line width=1.5+rand] (E) -- (MID);
                    \draw[line width=1.5+rand] (B) -- (NB);
                    \draw[line width=1.5+rand] (C) -- (CD) -- (D);
                    \draw[line width=1.5+rand] (J) -- node[midway] (JK) {} (K);
                    \draw[line width=1.5+rand] (MID) -- (NB) -- (RIGHT) -- (CD) -- (RIGHT) -- (JK);
                    \draw[line width=1.5+rand] (NB) -- (L);
                    \draw[line width=1.5+rand] (RIGHT) -- (I);
                \end{tikzpicture}
            \end{minipage}

            \vspace{0.5cm}

            \begin{minipage}[0.4\textheight]{0.1\textwidth}
                \begin{tikzpicture}[scale=0.34, every node/.style={scale=0.4, circle,fill=TUMGray}, draw=TUMGray]
                    \draw (-4.0,4.3) node[fill=TUMSecondaryBlue] (A) {};
                    \draw (-2.6,3.9) node[fill=TUMSecondaryBlue] (B) {};
                    \draw (-1.8,3.1) node[fill=TUMSecondaryBlue] (C) {};
                    \draw (-1.2,4.1) node[fill=TUMSecondaryBlue] (D) {};
                    \draw (-5.4,1.9) node[fill=TUMAccentOrange] (E) {};
                    \draw (-6.2,2.8) node[fill=TUMAccentOrange] (F) {};
                    \draw (-6.2,1.0) node[fill=TUMAccentOrange] (G) {};
                    \draw (-4.8,0.6) node[fill=TUMAccentOrange] (H) {};
                    \draw (-0.5,0.7) node[fill=TUMAccentGreen] (I) {};
                    \draw (-2.1,-0.4) node[fill=TUMAccentGreen] (J) {};
                    \draw (-0.4,-0.3) node[fill=TUMAccentGreen] (K) {};
                    \draw (-2.2,1.1) node[fill=TUMAccentGreen] (L) {};


                    \node at (-5,3.5) (AF) {};
                    \node at (-4,2)  (MID) {};
                    \node at (-3,2.25) (NB) {};
                    \node at (-1,2) (RIGHT) {};
                    \node at (-1,3) (CD) {};

                    \draw[line width=1.5+rand] (G) -- node[midway] (GH) {} (H);
                    \draw[line width=1.5+rand] (A) -- (AF) -- (F);
                    \draw[line width=1.5+rand] (GH) -- (MID) -- (AF);
                    \draw[line width=1.5+rand] (E) -- (MID);
                    \draw[line width=1.5+rand] (B) -- (NB);
                    \draw[line width=1.5+rand] (C) -- (CD) -- (D);
                    \draw[line width=1.5+rand] (J) -- node[midway] (JK) {} (K);
                    \draw[line width=1.5+rand] (MID) -- (NB) -- (RIGHT) -- (CD) -- (RIGHT) -- (JK);
                    \draw[line width=1.5+rand] (NB) -- (L);
                    \draw[line width=1.5+rand] (RIGHT) -- (I);

                    \draw[TUMAccentOrange, line width=1pt] (-7,-1.25) rectangle (0,5);
                \end{tikzpicture}	
            \end{minipage}

            \vspace{0.5cm}

            \begin{minipage}[0.4\textheight]{0.1\textwidth}
                \begin{tikzpicture}[scale=0.34, every node/.style={scale=0.4, circle,fill=TUMGray}, draw=TUMGray]
                    \draw (-4.0,4.3) node[fill=TUMAccentOrange] (A) {};
                    \draw (-2.6,3.9) node[fill=TUMSecondaryBlue] (B) {};
                    \draw (-1.8,3.1) node[fill=TUMSecondaryBlue] (C) {};
                    \draw (-1.2,4.1) node[fill=TUMSecondaryBlue] (D) {};
                    \draw (-5.4,1.9) node[fill=TUMSecondaryBlue] (E) {};
                    \draw (-6.2,2.8) node[fill=TUMAccentOrange] (F) {};
                    \draw (-6.2,1.0) node[fill=TUMAccentOrange] (G) {};
                    \draw (-4.8,0.6) node[fill=TUMAccentGreen] (H) {};
                    \draw (-0.5,0.7) node[fill=TUMAccentGreen] (I) {};
                    \draw (-2.1,-0.4) node[fill=TUMAccentOrange] (J) {};
                    \draw (-0.4,-0.3) node[fill=TUMAccentGreen] (K) {};
                    \draw (-2.2,1.1) node[fill=TUMAccentGreen] (L) {};

                    \node at (-5.5,4) (AF) {};
                    \node at (-4,3)  (MID) {};
                    \node at (-3,2.0) (NB) {};
                    \node at (-1.5,2) (RIGHT) {};
                    \node at (-0.5,3) (CD) {};

                    \draw[line width=2.0+rand] (G) -- node[midway] (GH) {} (H);
                    \draw[line width=2.0+rand] (A) -- (AF) -- (F);
                    \draw[line width=2.0+rand] (GH) -- (MID) -- (AF);
                    \draw[line width=2.0+rand] (E) -- (MID);
                    \draw[line width=2.0+rand] (B) -- (NB);
                    \draw[line width=2.0+rand] (C) -- (CD) -- (D);
                    \draw[line width=2.0+rand] (J) -- node[midway] (JK) {} (K);
                    \draw[line width=2.0+rand] (MID) -- (NB) -- (RIGHT) -- (CD) -- (RIGHT) -- (JK);
                    \draw[line width=2.0+rand] (NB) -- (L);
                    \draw[line width=2.0+rand] (RIGHT) -- (I);

                \end{tikzpicture}	
            \end{minipage}

            \vfill
            \hspace{0.5\textwidth} {\LARGE \vdots}
        \end{column}

        \begin{column}{0.1\textwidth}
            \begin{tikzpicture}
                \draw[->, line width=1pt] (0,0) -- node[above, midway, scale=0.6, yshift=3pt] {Anwenden} (1,0);
            \end{tikzpicture}
        \end{column}
        
        \begin{column}{0.24\textwidth}
            \begin{tikzpicture}[scale=0.4, every node/.style={scale=0.5, circle}]
                \drawgraph
            \end{tikzpicture}	
        \end{column}
    \end{columns}
\end{frame}

\begin{frame}[fragile]{Berechnung der Signaturen}
    \begin{center}
        \begin{tikzpicture}[
            mynode/.style={circle, scale=0.6, fill=black!40},
            randomdraw/.style={decoration={random steps, segment length=5pt, amplitude=2.3pt}}
        ]
            \foreach \x/\y/\lbl in {%
                -3.5/0/A,
                -1/0/W,
                1/0/V,
                3.5/0/B,
                0/1/R,
                0.5/-0.75/C,
                1.5/-0.75/U%
            }{
                \node at (\x,\y) [mynode] (\lbl) {};
            }
            
            \foreach \lbl in {A,W,V,B}{
                \path (R) edge (\lbl);
            }
            
            \path (A) -- node[font=\Large]{\ldots} (W);
            \path (V) -- node[font=\Large]{\ldots} (B);
            \path (C) -- node[font=\large]{\ldots} (U);
            
            \draw (C) -- (V) -- (U);
            
            \draw (R) -- (0,1.5);
            
            \foreach \lbl/\lft/\rgt in {U/1.25/1.75,C/0.25/0.75,A/-4/-3,B/3/4,W/-1.5/-0.5}{%
                \path[draw] (\lbl) -- (\lft, -1.75) -- coordinate[midway](\lbl{}M) (\rgt, -1.75) -- (\lbl); 
            }
            
            \foreach \lbl/\n in {W/$w$,V/$v$,U/$u$}{%
                \node [label={0:\large\n}] at (\lbl) {};
            }
            
            \draw[rounded corners] (-4.3,-2.05) rectangle (2.05,0.3);
            \node at (-3.9, 0.55) {\large$L_v$};
            
            \pgfmathsetseed{1}
            \foreach \lbl/\ln/\ls/\rn/\rs/\out/\in in {
                A/-4/near end/-3/midway/30/120,
                W/-1.5/midway/-0.5/near end/20/150,
                C/0.25/midway/0.75/near end/30/120,
                U/1.25/near end/1.75/midway/30/160%
            } {
            
                \path (\lbl) -- coordinate[\ls](\lbl{}L) (\ln, -1.75);
                \path (\lbl) -- coordinate[\rs](\lbl{}R) (\rn, -1.75);
                \draw[fill=black] decorate[randomdraw]{(\lbl{}L) --  (\lbl{}R)} -- (\rn, -1.75) -- (\ln, -1.75) -- cycle; 
            }
            
            
            \draw (-3.5,-2.25) -- node[midway, label={270:\large$m$}](M){} (1.5, -2.25);
            \draw (M.north) -- (M.south);
            \foreach \x in {-3.5,-1,0.5,1.5}{%
                \draw (\x, -1.75) -- (\x, -2.25);
            }
        \end{tikzpicture}
    \end{center}

    \only<2>{
       \begin{equation*}
           \begin{aligned}
               C_v(\vec{g}, m) = \min \{ & C_w(\vec{g}_w, m - x) + C_u(\vec{g}_u, x) \mid \\ & \qquad 0 \leq x \leq m \land \vec{g}_w + \vec{g}_u = \vec{g} \}
           \end{aligned}
       \end{equation*}
    }   

    \only<3>{
        \begin{equation*}
            \begin{aligned}
                C_v(\vec{g}, m) = \weight(e) + \min \{ & C_w(\vec{g}_w, m - n_v) + C_u(\vec{g}_u, n_v - x) \mid \\ & \qquad 1 \leq x \leq \mu \land \vec{g}_w + \vec{g}_u + \vec{e}(x) = \vec{g} \}
            \end{aligned}
        \end{equation*}
    }
\end{frame}


\section{Implementierung}
\begin{frame}{Menge von Äquivalenzklassen}
    \begin{itemize}[<+(1)->]
        \item Menge von Äquivalenzklassen als Hashmap
        \item Hashfunktion von Boost für Signaturen 
        \item Großer Einfluss auf die Laufzeit: Verbesserungen?
    \end{itemize}
\end{frame}


\section{Experimentelle Evaluation}
\subsection{Laufzeit}
\begin{frame}{Methodik}
    \begin{itemize}[<+(1)->]
        \item Zufallsbäume: Binärbäume, Preferential Attachment
        \item 20 Versuche pro Parameterkombination
        \item Visualisierung mit Boxplots
    \end{itemize}
\end{frame}

\begin{frame}{Laufzeit in Abhängigkeit von $n$}
    \begin{table}
        \centering
        \begin{tabular}{l*{4}{c}}
            \toprule
            & 100 & 200 & 400 & 800 \\
            \midrule
            Preferential Attachment & 49 & 452 & 9741 & 160347 \\
            Binärbaum & 132 & 2228 & 37758 & 644216 \\ 
            \bottomrule
        \end{tabular}
    \end{table}
\end{frame}

\begin{frame}[fragile]{Laufzeit in Abhängigkeit von $\eps$}
    \begin{center}
        \begin{tikzpicture}[font=\footnotesize]
            \renewcommand{\datadir}{data/trees/runtime/imbalance}
            \renewcommand{\plotdata}[2]{%
                \addplot+ [boxplot] table[y=#1] {\datadir/#2/n50k2.dat};%
                \addplot+ [boxplot] table[y=#1] {\datadir/#2/n50k4.dat};%
                \addplot+ [boxplot] table[y=#1] {\datadir/#2/n100k2.dat};%
            }

            \pgfplotsset{
                boxplot/draw direction=y,
                boxplot={
                    draw position={1/4 + floor(\plotnumofactualtype/3) + 1/4*mod(\plotnumofactualtype,3)},
                    box extend=0.2,
                },
                xtick={0,1,2,...,10},
                x tick label as interval,
                x tick label style={
                    text width=2.5cm,
                    align=center
                }
            }

            % Useful links: 
            % https://tex.stackexchange.com/questions/37568/colors-and-legend-in-groupplots-barplot
            % https://tex.stackexchange.com/questions/183778/how-can-several-boxplots-be-combined-into-a-single-figure-as-groups
            \begin{groupplot}[
                group style={
                    group size=2 by 1,
                    ylabels at=edge left,
                    xlabels at=edge bottom,
                    xticklabels at=edge bottom,
                    horizontal sep=1.5cm
                },
                xtick pos=bottom,
                width=0.5*\linewidth,
                height=6cm,
                cycle list={TUMBlue, TUMAccentOrange, TUMAccentGreen},
                xticklabels={0.35, 0.3, 0.25, 0.2}
            ]

                \nextgroupplot[
                    title=Preferential Attachment,
                    ymax=2700,
                    ytick distance=500,
                    ymajorgrids,
                    legend to name=grouplegend,
                    legend columns=-1,
                    legend style={/tikz/every even column/.append style={column sep=5pt}}
                ]
                \addplot+ [boxplot] table[y=7/20] {\datadir/pref_attach/n50k2.dat};%
                \addlegendentry{{$n=50$, $k=2$}}
                \addplot+ [boxplot] table[y=7/20] {\datadir/pref_attach/n50k4.dat};%
                \addlegendentry{{$n=50$, $k=4$}}
                \addplot+ [boxplot] table[y=7/20] {\datadir/pref_attach/n100k2.dat};%
                \addlegendentry{{$n=100$, $k=2$}}
                \pgfplotsinvokeforeach {3/10, 5/20, 2/10} {
                    \plotdata{#1}{pref_attach}
                }

                \nextgroupplot[
                    title=Binärbaum, ymax=6300,
                    ytick distance=1000,
                    ymajorgrids
                ]
                \pgfplotsinvokeforeach {7/20, 3/10, 5/20, 2/10} {
                    \plotdata{#1}{bnary}
                }
            \end{groupplot}
            \path (group c1r1.south east) -- node[below, yshift=-1cm]{\small \pgfplotslegendfromname{grouplegend}} (group c2r1.south west);
        \end{tikzpicture}
    \end{center}
\end{frame}

\begin{frame}[fragile]{Laufzeit in Abhängigkeit von $k$}
    \begin{center}
        \begin{tikzpicture}[font=\footnotesize]
            \renewcommand{\datadir}{data/trees/runtime/kparts}
            \renewcommand{\plotdata}[2]{%
                \addplot+ [boxplot] table[y=#1] {\datadir/#2/n50i1div3.dat};%
                \addplot+ [boxplot] table[y=#1] {\datadir/#2/n50i1div4.dat};%
                \addplot+ [boxplot] table[y=#1] {\datadir/#2/n60i1div3.dat};%
            }

            \pgfplotsset{
                boxplot/draw direction=y,
                boxplot={
                    draw position={1/4 + floor(\plotnumofactualtype/3) + 1/4*mod(\plotnumofactualtype,3)},
                    box extend=0.2,
                },
                xtick={0,1,2,...,10},
                x tick label as interval,
                x tick label style={
                    text width=2.5cm,
                    align=center
                }
            }

            % Useful links: 
            % https://tex.stackexchange.com/questions/37568/colors-and-legend-in-groupplots-barplot
            % https://tex.stackexchange.com/questions/183778/how-can-several-boxplots-be-combined-into-a-single-figure-as-groups
            \begin{groupplot}[
                group style={
                    group size=2 by 1,
                    ylabels at=edge left,
                    xlabels at=edge bottom,
                    xticklabels at=edge bottom,
                    horizontal sep=1.5cm
                },
                xtick pos=bottom,
                width=0.5*\linewidth,
                height=6cm,
                cycle list={TUMBlue, TUMAccentOrange, TUMAccentGreen},
                xticklabels={3,6,8,10}
            ]

                \nextgroupplot[
                    title=Preferential Attachment,
                    ymax=1900,
                    ytick distance=500,
                    ymajorgrids,
                    legend to name=grouplegend,
                    legend columns=-1,
                    legend style={/tikz/every even column/.append style={column sep=5pt}}
                ]
                \addplot+ [boxplot] table[y=3] {\datadir/pref_attach/n50i1div3.dat};%
                \addlegendentry{{$n=50$, $\eps=1/3$}}
                \addplot+ [boxplot] table[y=3] {\datadir/pref_attach/n50i1div4.dat};%
                \addlegendentry{{$n=50$, $\eps=1/4$}}
                \addplot+ [boxplot] table[y=3] {\datadir/pref_attach/n60i1div3.dat};%
                \addlegendentry{{$n=60$, $\eps=1/3$}}
                \pgfplotsinvokeforeach {6,8,10} {
                    \plotdata{#1}{pref_attach}
                }

                \nextgroupplot[
                    title=Binärbaum, ymax=7900,
                    ytick distance=2000,
                    ymajorgrids
                ]
                \pgfplotsinvokeforeach {3,6,8,10} {
                    \plotdata{#1}{bnary}
                }
            \end{groupplot}
            \path (group c1r1.south east) -- node[below, yshift=-1cm]{\small \pgfplotslegendfromname{grouplegend}} (group c2r1.south west);
        \end{tikzpicture}
    \end{center}
\end{frame}

\begin{frame}{Exklusive obere Schranken für $n=50$ und $\eps=1/4$}
    \begin{table}
        \centering
        \begin{tabular}{l*{11}{c}}
            \toprule
            & \multicolumn{9}{c}{Signatur $\vec{g}$} \\
            $k$ & $g_0$ & $g_1$ & $g_2$ & $g_3$ & $g_4$ & $g_5$ & $g_6$ & $g_7$ & $g_8$ \\
            \midrule
            3 & 5 & 6 & 7 & 9 & 11 & 13 & 17 & 21 & 22 \\
            4 & 4 & 5 & 6 & 7 & 8 & 10 & 13 & 16 & 17 \\
            5 & 3 & \alert{4} & \alert{4} & 5 & 7 & 8 & 10 & 12 & 13 \\ 
            6 & \alert{3} & \alert{3} & 4 & 5 & 6 & 7 & 9 & 11 & 12 \\ 
            8 & 2 & \alert{3} & \alert{3} & 4 & 5 & 6 & 7 & 9 & 9 \\ 
            10 & \alert{2} & \alert{2} & \alert{2} & 3 & \alert{4} & \alert{4} & 5 & 6 & 7 \\ 
            \bottomrule
        \end{tabular}
    \end{table}
\end{frame}

\subsection{Lösungsqualität auf Bäumen}

\begin{frame}[fragile]{Lösungsqualität für Pref. Attach. abhängig von $n$}
    \begin{center}
        \begin{tikzpicture}[font=\footnotesize]
            \renewcommand{\datadir}{data/trees/quality/node_count}
            \renewcommand{\plotdata}[2]{%
                \addplot+ [boxplot] table[y=#1] {\datadir/#2_pref_attach_k2i1div3.dat};%
                \addplot+ [boxplot] table[y=#1] {\datadir/#2_pref_attach_k4i1div3.dat};%
                \addplot+ [boxplot] table[y=#1] {\datadir/#2_pref_attach_k6i1div3.dat};%
                }
            
            \pgfplotsset{
                boxplot/draw direction=y,
                boxplot={
                    draw position={1/4 + floor(\plotnumofactualtype/3) + 1/4*mod(\plotnumofactualtype,3)},
                    box extend=0.2,
                    whisker range=10
                },
                xtick={0,1,2,...,10},
                x tick label as interval,
                x tick label style={
                    text width=2.5cm,
                    align=center
                }
            }
            
            % Useful links: 
            % https://tex.stackexchange.com/questions/37568/colors-and-legend-in-groupplots-barplot
            % https://tex.stackexchange.com/questions/183778/how-can-several-boxplots-be-combined-into-a-single-figure-as-groups
            \begin{groupplot}[
                    group style={
                        group size=2 by 2,
                        ylabels at=edge left,
                        xlabels at=edge bottom,
                        horizontal sep=1.25cm,
                        vertical sep=1.25cm,
                    },
                    xtick pos=bottom,
                    height=4.3cm,
                    width=0.5*\linewidth,
                    cycle list={TUMBlue, TUMAccentOrange, TUMAccentGreen},
                    xticklabels={70,80,90,100},
                    ymax=1.3,
                    ymin=-0.15,
                    ytick distance=0.2,
                    ymajorgrids
                ]
        
                \nextgroupplot[
                    title=METIS recursive,
                    legend to name=grouplegend,
                    legend columns=1,
                    legend style={/tikz/every even column/.append style={column sep=5pt}}
                ]
                \addplot+ [boxplot] table[y=70] {\datadir/METIS_Recursive_pref_attach_k2i1div3.dat};%
                \addlegendentry{{$k=2$, $\eps=1/3$}}
                \addplot+ [boxplot] table[y=70] {\datadir/METIS_Recursive_pref_attach_k4i1div3.dat};%
                \addlegendentry{{$k=4$, $\eps=1/3$}}
                \addplot+ [boxplot] table[y=70] {\datadir/METIS_Recursive_pref_attach_k6i1div3.dat};%
                \addlegendentry{{$k=6$, $\eps=1/3$}}
                \pgfplotsinvokeforeach {80,90,100} {
                    \plotdata{#1}{METIS_Recursive}
                }
        
                \nextgroupplot[
                    title=METIS k-way
                ]
                \pgfplotsinvokeforeach {70,80,90,100} {
                    \plotdata{#1}{METIS_Kway}
                }
        
                \nextgroupplot[
                    title=KaFFPa
                ]
                \pgfplotsinvokeforeach {70,80,90,100} {
                    \plotdata{#1}{KaFFPa}
                }
                
            \end{groupplot}
            \path (group c1r2.east) -- node[below, xshift=1.5cm]{\small \pgfplotslegendfromname{grouplegend}} (group c2r1.south);
        
        \end{tikzpicture}
    \end{center}
\end{frame}

\begin{frame}[fragile]{Lösungsqualität für Binärbäume abhängig von $n$}
    \begin{center}
        \begin{tikzpicture}[font=\footnotesize]
            \renewcommand{\datadir}{data/trees/quality/node_count}
            \renewcommand{\plotdata}[2]{%
                \addplot+ [boxplot] table[y=#1] {\datadir/#2_bnary_k2i1div3.dat};%
                \addplot+ [boxplot] table[y=#1] {\datadir/#2_bnary_k4i1div3.dat};%
                \addplot+ [boxplot] table[y=#1] {\datadir/#2_bnary_k6i1div3.dat};%
                }
            
            \pgfplotsset{
                boxplot/draw direction=y,
                boxplot={
                    draw position={1/4 + floor(\plotnumofactualtype/3) + 1/4*mod(\plotnumofactualtype,3)},
                    box extend=0.2,
                    whisker range=10
                },
                xtick={0,1,2,...,10},
                x tick label as interval,
                x tick label style={
                    text width=2.5cm,
                    align=center
                }
            }
            
            % Useful links: 
            % https://tex.stackexchange.com/questions/37568/colors-and-legend-in-groupplots-barplot
            % https://tex.stackexchange.com/questions/183778/how-can-several-boxplots-be-combined-into-a-single-figure-as-groups
            \begin{groupplot}[
                    group style={
                        group size=2 by 2,
                        ylabels at=edge left,
                        xlabels at=edge bottom,
                        horizontal sep=1.25cm,
                        vertical sep=1.25cm,
                    },
                    xtick pos=bottom,
                    height=4.3cm,
                    width=0.5*\linewidth,
                    cycle list={TUMBlue, TUMAccentOrange, TUMAccentGreen},
                    xticklabels={70,80,90,100},
                    ymax=1.3,
                    ymin=-0.15,
                    ytick distance=0.2,
                    ymajorgrids
                ]
        
                \nextgroupplot[
                    title=METIS recursive,
                    legend to name=grouplegend,
                    legend columns=1,
                    legend style={/tikz/every even column/.append style={column sep=5pt}}
                ]
                \addplot+ [boxplot] table[y=70] {\datadir/METIS_Recursive_bnary_k2i1div3.dat};%
                \addlegendentry{{$k=2$, $\eps=1/3$}}
                \addplot+ [boxplot] table[y=70] {\datadir/METIS_Recursive_bnary_k4i1div3.dat};%
                \addlegendentry{{$k=4$, $\eps=1/3$}}
                \addplot+ [boxplot] table[y=70] {\datadir/METIS_Recursive_bnary_k6i1div3.dat};%
                \addlegendentry{{$k=6$, $\eps=1/3$}}
                \pgfplotsinvokeforeach {80,90,100} {
                    \plotdata{#1}{METIS_Recursive}
                }
        
                \nextgroupplot[
                    title=METIS k-way
                ]
                \pgfplotsinvokeforeach {70,80,90,100} {
                    \plotdata{#1}{METIS_Kway}
                }
        
                \nextgroupplot[
                    title=KaFFPa
                ]
                \pgfplotsinvokeforeach {70,80,90,100} {
                    \plotdata{#1}{KaFFPa}
                }
                
            \end{groupplot}
            \path (group c1r2.east) -- node[below, xshift=1.5cm]{\small \pgfplotslegendfromname{grouplegend}} (group c2r1.south);
        
        \end{tikzpicture}
    \end{center}
\end{frame}

\begin{frame}[fragile]{Lösungsqualität für Pref. Attach. abhängig von $\eps$}
    \begin{center}
        \begin{tikzpicture}[font=\footnotesize]
            \renewcommand{\datadir}{data/trees/quality/imbalance}
            \renewcommand{\plotdata}[2]{%
                \addplot+ [boxplot] table[y=#1] {\datadir/#2_pref_attach_n80k2.dat};%
                \addplot+ [boxplot] table[y=#1] {\datadir/#2_pref_attach_n80k3.dat};%
                \addplot+ [boxplot] table[y=#1] {\datadir/#2_pref_attach_n80k4.dat};%
                }
            
            \pgfplotsset{
                boxplot/draw direction=y,
                boxplot={
                    draw position={1/4 + floor(\plotnumofactualtype/3) + 1/4*mod(\plotnumofactualtype,3)},
                    box extend=0.2,
                    whisker range=10
                },
                xtick={0,1,2,...,10},
                x tick label as interval,
                x tick label style={
                    text width=2.5cm,
                    align=center
                }
            }
            
            % Useful links: 
            % https://tex.stackexchange.com/questions/37568/colors-and-legend-in-groupplots-barplot
            % https://tex.stackexchange.com/questions/183778/how-can-several-boxplots-be-combined-into-a-single-figure-as-groups
            \begin{groupplot}[
                    group style={
                        group size=2 by 2,
                        ylabels at=edge left,
                        xlabels at=edge bottom,
                        horizontal sep=1.25cm,
                        vertical sep=1.25cm,
                    },
                    xtick pos=bottom,
                    height=4.3cm,
                    width=0.5*\linewidth,
                    cycle list={TUMBlue, TUMAccentOrange, TUMAccentGreen},
                    xticklabels={0.4,0.35,0.3,0.25},
                    ymax=1.6,
                    ymin=-0.15,
                    ytick distance=0.25,
                    ymajorgrids
                ]
        
                \nextgroupplot[
                    title=METIS recursive,
                    legend to name=grouplegend,
                    legend columns=1,
                    legend style={/tikz/every even column/.append style={column sep=5pt}}
                ]
                \addplot+ [boxplot] table[y=8/20] {\datadir/METIS_Recursive_pref_attach_n80k2.dat};%
                \addlegendentry{{$n=80$, $k=2$}}
                \addplot+ [boxplot] table[y=8/20] {\datadir/METIS_Recursive_pref_attach_n80k3.dat};%
                \addlegendentry{{$n=80$, $k=3$}}
                \addplot+ [boxplot] table[y=8/20] {\datadir/METIS_Recursive_pref_attach_n80k4.dat};%
                \addlegendentry{{$n=80$, $k=4$}}
                \pgfplotsinvokeforeach {7/20,6/20,5/20} {
                    \plotdata{#1}{METIS_Recursive}
                }
        
                \nextgroupplot[
                    title=METIS k-way
                ]
                \pgfplotsinvokeforeach {8/20,7/20,6/20,5/20} {
                    \plotdata{#1}{METIS_Kway}
                }
        
                \nextgroupplot[
                    title=KaFFPa
                ]
                \pgfplotsinvokeforeach {8/20,7/20,6/20,5/20} {
                    \plotdata{#1}{KaFFPa}
                }
                
            \end{groupplot}
            \path (group c1r2.east) -- node[below, xshift=1.5cm]{\small \pgfplotslegendfromname{grouplegend}} (group c2r1.south);
        
        \end{tikzpicture}
    \end{center}
\end{frame}

\begin{frame}[fragile]{Lösungsqualität für Binärbäume abhängig von $\eps$}
    \begin{center}
        \begin{tikzpicture}[font=\footnotesize]
            \renewcommand{\datadir}{data/trees/quality/imbalance}
            \renewcommand{\plotdata}[2]{%
                \addplot+ [boxplot] table[y=#1] {\datadir/#2_bnary_n80k2.dat};%
                \addplot+ [boxplot] table[y=#1] {\datadir/#2_bnary_n80k3.dat};%
                \addplot+ [boxplot] table[y=#1] {\datadir/#2_bnary_n80k4.dat};%
                }
            
            \pgfplotsset{
                boxplot/draw direction=y,
                boxplot={
                    draw position={1/4 + floor(\plotnumofactualtype/3) + 1/4*mod(\plotnumofactualtype,3)},
                    box extend=0.2,
                    whisker range=10
                },
                xtick={0,1,2,...,10},
                x tick label as interval,
                x tick label style={
                    text width=2.5cm,
                    align=center
                }
            }
            
            % Useful links: 
            % https://tex.stackexchange.com/questions/37568/colors-and-legend-in-groupplots-barplot
            % https://tex.stackexchange.com/questions/183778/how-can-several-boxplots-be-combined-into-a-single-figure-as-groups
            \begin{groupplot}[
                    group style={
                        group size=2 by 2,
                        ylabels at=edge left,
                        xlabels at=edge bottom,
                        horizontal sep=1.25cm,
                        vertical sep=1.25cm,
                    },
                    xtick pos=bottom,
                    height=4.3cm,
                    width=0.5*\linewidth,
                    cycle list={TUMBlue, TUMAccentOrange, TUMAccentGreen},
                    xticklabels={0.4,0.35,0.3,0.25},
                    ymax=1.4,
                    ymin=-0.15,
                    ytick distance=0.25,
                    ymajorgrids
                ]
        
                \nextgroupplot[
                    title=METIS recursive,
                    legend to name=grouplegend,
                    legend columns=1,
                    legend style={/tikz/every even column/.append style={column sep=5pt}},
                ]
                \addplot+ [boxplot] table[y=8/20] {\datadir/METIS_Recursive_bnary_n80k2.dat};%
                \addlegendentry{{$n=80$, $k=2$}}
                \addplot+ [boxplot] table[y=8/20] {\datadir/METIS_Recursive_bnary_n80k3.dat};%
                \addlegendentry{{$n=80$, $k=3$}}
                \addplot+ [boxplot] table[y=8/20] {\datadir/METIS_Recursive_bnary_n80k4.dat};%
                \addlegendentry{{$n=80$, $k=4$}}
                \pgfplotsinvokeforeach {7/20,6/20,5/20} {
                    \plotdata{#1}{METIS_Recursive}
                }
        
                \nextgroupplot[
                    title=METIS k-way
                ]
                \pgfplotsinvokeforeach {8/20,7/20,6/20,5/20} {
                    \plotdata{#1}{METIS_Kway}
                }
        
                \nextgroupplot[
                    title=KaFFPa
                ]
                \pgfplotsinvokeforeach {8/20,7/20,6/20,5/20} {
                    \plotdata{#1}{KaFFPa}
                }
                
            \end{groupplot}
            \path (group c1r2.east) -- node[below, xshift=1.5cm]{\small \pgfplotslegendfromname{grouplegend}} (group c2r1.south);
        
        \end{tikzpicture}
    \end{center}
\end{frame}

\subsection{Reale Daten}

\begin{frame}{Reale Datensätze}
\begin{table}
    \centering
    \begin{tabular}{lccc}
        \toprule
        Datensatz & Knoten & Kanten & Anzahl der Dreiecke \\
        \midrule
        as19990829 & 103 & 248 & 228 \\
        email-Eu-core & 1005 & 25571 & 105461 \\
        ca-GrQc & 5242 & 14496 & 48260 \\
        ego-Facebook & 4039 & 88234 & 1612010 \\
        \bottomrule
    \end{tabular}
\end{table}
\end{frame}

\begin{frame}[fragile]{as19990829}
    \begin{center}
        \begin{tikzpicture}[font=\footnotesize]
            \renewcommand{\datadir}{data/graphs/data_sets/as}
            \renewcommand{\datafile}[1]{\datadir/#1_as19990829.dat}

            \pgfplotsset{
                bar cycle list/.style={
                    cycle list={{fill=TUMBlue}, {fill=TUMAccentOrange}, {fill=TUMAccentGreen}, {fill=TUMSecondaryBlue2}}
                }
            }

            % https://tex.stackexchange.com/questions/208102/fix-group-bar-chart-spacing
            % https://tex.stackexchange.com/questions/101320/grouped-bar-chart
            \begin{groupplot}[
                group style={
                    group size=2 by 2,
                    vertical sep=1.5cm,
                    ylabels at=edge left
                },
                xticklabels={0.4,0.35,0.3,0.25},
                ymajorgrids,
                xtick pos=bottom,
                xtick=data,
                enlarge x limits=0.25,
                width=0.5*\linewidth,
                height=4cm,
                ymax=1.9,
                ytick distance=0.5,
                major x tick style=transparent,
                extra y ticks={1},
                extra y tick style={grid=major},
                ymin=0,
                /pgf/bar width=2.5pt,
            ]

                \nextgroupplot[
                    title=METIS recursive,
                    legend to name=grouplegend,
                    legend columns=2,
                    legend style={/tikz/every even column/.append style={column sep=5pt}},
                    ybar
                ]
                \pgfplotsinvokeforeach {2,3,4,6}{
                    \addplot+ table [x expr=\coordindex, y=#1] {\datafile{METIS_Recursive}};
                \addlegendentry{{$k=#1$}}
                }			

                \nextgroupplot[
                    title=METIS k-way,
                    ybar
                ]
                \pgfplotsinvokeforeach {2,3,4,6}{
                    \addplot+ table [x expr=\coordindex, y=#1] {\datafile{METIS_Kway}};
                }			

                \nextgroupplot[
                    title=KaFFPa,
                    ybar
                ]
                \pgfplotsinvokeforeach {2,3,4,6}{
                    \addplot+ table [x expr=\coordindex, y=#1] {\datafile{KaFFPa}};
                }			
            \end{groupplot}
            \path (group c1r2.east) -- node[xshift=0.25*\linewidth]{\small \pgfplotslegendfromname{grouplegend}} (group c1r2.east);
        \end{tikzpicture}
    \end{center}
\end{frame}

\begin{frame}[fragile]{email-Eu-core}
    \begin{center}
        \begin{tikzpicture}[font=\footnotesize]
            \renewcommand{\datadir}{data/graphs/data_sets/email}
            \renewcommand{\datafile}[1]{\datadir/#1_email-Eu-core.dat}

            \pgfplotsset{
                bar cycle list/.style={
                    cycle list={{fill=TUMBlue}, {fill=TUMAccentOrange}, {fill=TUMAccentGreen}, {fill=TUMSecondaryBlue2}}
                }
            }

            % https://tex.stackexchange.com/questions/208102/fix-group-bar-chart-spacing
            % https://tex.stackexchange.com/questions/101320/grouped-bar-chart
            \begin{groupplot}[
                group style={
                    group size=2 by 2,
                    vertical sep=1.5cm,
                    ylabels at=edge left
                },
                xticklabels={0.4,0.35,0.3,0.25},
                ymajorgrids,
                xtick pos=bottom,
                xtick=data,
                enlarge x limits=0.25,
                width=0.5*\linewidth,
                height=4cm,
                ymax=1.9,
                ytick distance=0.5,
                major x tick style=transparent,
                ymin=0,
                /pgf/bar width=2.5pt,
            ]

                \nextgroupplot[
                    title=METIS recursive,
                    legend to name=grouplegend,
                    legend columns=2,
                    legend style={/tikz/every even column/.append style={column sep=5pt}},
                    ybar
                ]
                \pgfplotsinvokeforeach {2,3,4,6}{
                    \addplot+ table [x expr=\coordindex, y=#1] {\datafile{METIS_Recursive}};
                \addlegendentry{{$k=#1$}}
                }			

                \nextgroupplot[
                    title=METIS k-way,
                    ybar
                ]
                \pgfplotsinvokeforeach {2,3,4,6}{
                    \addplot+ table [x expr=\coordindex, y=#1] {\datafile{METIS_Kway}};
                }			

                \nextgroupplot[
                    title=KaFFPa,
                    ybar
                ]
                \pgfplotsinvokeforeach {2,3,4,6}{
                    \addplot+ table [x expr=\coordindex, y=#1] {\datafile{KaFFPa}};
                }			
            \end{groupplot}
            \path (group c1r2.east) -- node[xshift=0.25*\linewidth]{\small \pgfplotslegendfromname{grouplegend}} (group c1r2.east);
        \end{tikzpicture}
    \end{center}
\end{frame}

\begin{frame}[fragile]{ca-GrQc}
    \begin{center}
        \begin{tikzpicture}[font=\footnotesize]
            \renewcommand{\datadir}{data/graphs/data_sets/caGrQc}
            \renewcommand{\datafile}[1]{\datadir/#1_ca-GrQc.dat}

            \pgfplotsset{
                bar cycle list/.style={
                    cycle list={{fill=TUMBlue}, {fill=TUMAccentOrange}, {fill=TUMAccentGreen}}
                }
            }

            % https://tex.stackexchange.com/questions/208102/fix-group-bar-chart-spacing
            % https://tex.stackexchange.com/questions/101320/grouped-bar-chart
            \begin{groupplot}[
                group style={
                    group size=2 by 2,
                    vertical sep=1.5cm,
                    ylabels at=edge left
                },
                xticklabels={0.4,0.35,0.3,0.25},
                ymajorgrids,
                xtick pos=bottom,
                xtick=data,
                enlarge x limits=0.25,
                width=0.5*\linewidth,
                height=4cm,
                ymax=3.3,
                ytick distance=1.0,
                major x tick style=transparent,
                ymin=0,
                /pgf/bar width=2.5pt,
            ]

                \nextgroupplot[
                    title=METIS recursive,
                    legend to name=grouplegend,
                    legend columns=2,
                    legend style={/tikz/every even column/.append style={column sep=5pt}},
                    ybar
                ]
                \pgfplotsinvokeforeach {2,3,4}{
                    \addplot+ table [x expr=\coordindex, y=#1] {\datafile{METIS_Recursive}};
                \addlegendentry{{$k=#1$}}
                }			

                \nextgroupplot[
                    title=METIS k-way,
                    ybar
                ]
                \pgfplotsinvokeforeach {2,3,4}{
                    \addplot+ table [x expr=\coordindex, y=#1] {\datafile{METIS_Kway}};
                }			

                \nextgroupplot[
                    title=KaFFPa,
                    ybar
                ]
                \pgfplotsinvokeforeach {2,3,4}{
                    \addplot+ table [x expr=\coordindex, y=#1] {\datafile{KaFFPa}};
                }			
            \end{groupplot}
            \path (group c1r2.east) -- node[xshift=0.25*\linewidth]{\small \pgfplotslegendfromname{grouplegend}} (group c1r2.east);
        \end{tikzpicture}
    \end{center}
\end{frame}

\begin{frame}[fragile]{ego-Facebook}
    \begin{center}
        \begin{tikzpicture}[font=\footnotesize]
            \renewcommand{\datadir}{data/graphs/data_sets/facebook}
            \renewcommand{\datafile}[1]{\datadir/#1_ego-Facebook.dat}

            \pgfplotsset{
                bar cycle list/.style={
                    cycle list={{fill=TUMBlue}, {fill=TUMAccentOrange}, {fill=TUMAccentGreen}}
                }
            }

            % https://tex.stackexchange.com/questions/208102/fix-group-bar-chart-spacing
            % https://tex.stackexchange.com/questions/101320/grouped-bar-chart
            \begin{groupplot}[
                group style={
                    group size=2 by 2,
                    vertical sep=1.5cm,
                    ylabels at=edge left
                },
                xticklabels={0.4,0.35,0.3,0.25},
                ymajorgrids,
                xtick pos=bottom,
                xtick=data,
                enlarge x limits=0.25,
                width=0.5*\linewidth,
                height=4cm,
                ymax=12,
                ytick distance=2.0,
                major x tick style=transparent,
                extra y ticks={1},
                extra y tick labels={},
                ymin=0,
                /pgf/bar width=2.5pt,
            ]

                \nextgroupplot[
                    title=METIS recursive,
                    legend to name=grouplegend,
                    legend columns=2,
                    legend style={/tikz/every even column/.append style={column sep=5pt}},
                    ybar
                ]
                \pgfplotsinvokeforeach {2,3,4}{
                    \addplot+ table [x expr=\coordindex, y=#1] {\datafile{METIS_Recursive}};
                \addlegendentry{{$k=#1$}}
                }			

                \nextgroupplot[
                    title=METIS k-way,
                    ybar
                ]
                \pgfplotsinvokeforeach {2,3,4}{
                    \addplot+ table [x expr=\coordindex, y=#1] {\datafile{METIS_Kway}};
                }			

                \nextgroupplot[
                    title=KaFFPa,
                    ybar
                ]
                \pgfplotsinvokeforeach {2,3,4}{
                    \addplot+ table [x expr=\coordindex, y=#1] {\datafile{KaFFPa}};
                }			
            \end{groupplot}
            \path (group c1r2.east) -- node[xshift=0.25*\linewidth]{\small \pgfplotslegendfromname{grouplegend}} (group c1r2.east);
        \end{tikzpicture}
    \end{center}
\end{frame}


\begin{frame}{Fazit}
    \begin{itemize}[<+(1)->]
        \item Hohe Laufzeit selbst bei kleinen Graphen
        \item Gute Lösungsqualität
        \item Aber: KaFFPa trotzdem kompetitiv
    \end{itemize}
\end{frame}

\begin{frame}[standout]
    \LARGE Fragen?
\end{frame}

\end{document}
